%% The following is a directive for TeXShop to indicate the main file
%%!TEX root = diss.tex

\chapter{Abstract}

Psychophysical studies of perceptual learning find that perceivers only improve the accuracy of their perception on stimuli similar to what they were trained on.
In contrast, speech perception studies of perceptual learning find generalization to novel contexts when words contain a modified ambiguous sound.
This dissertation seeks to resolve the apparent conflict between the findings of these two research paradigms through incorporating attentional sets.
Attention can be oriented towards comprehension of the speaker's intended meaning or towards perception of a speaker's pronunciation.
Attention is proposed to affect perceptual learning as follows.
When attention is oriented towards comprehension, more abstract and less context-dependent representations are updated and the perceiver shows generalized perceptual learning, as seen in the speech perception literature.
When attention is oriented towards perception, more finely detailed and more context-dependent representations are updated and the perceiver shows less generalized perceptual learning, similar to what is seen in the psychophysics literature.
This proposal is supported by three experiments.
The first two implement a standard paradigm for perceptual learning in speech perception.
In these experiments, promoting a more perception-oriented attentional set causes less generalized perceptual learning.
The final experiment uses a novel paradigm where modified sounds are embedded in sentences during exposure.
Perceptual learning is found only when the modified sound is embedded in words that are not predictable from the sentence.
When modified sounds are in predictable words, no perceptual learning is observed.
To account for this lack of perceptual learning, I hypothesize that sounds in predictable sentence are less reliable than sounds in words in isolation or unpredictable sentences.
In the cases where perceptual learning is present, contexts which support comprehension-oriented attentional sets show larger perceptual learning effects than contexts promoting perception-oriented attentional sets.
I argue that attention is a key component to the generalization of perceptual learning to new contexts.


% Consider placing version information if you circulate multiple drafts
%\vfill
%\begin{center}
%\begin{sf}
%\fbox{Revision: \today}
%\end{sf}
%\end{center}
