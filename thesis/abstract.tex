%% The following is a directive for TeXShop to indicate the main file
%%!TEX root = diss.tex

\chapter{Abstract}

Psychophysical studies of perceptual learning find that perceivers only improve their perception on stimuli similar to what they were trained on.
In contrast, speech perception studies of perceptual learning find generalization to novel contexts when words contain a modified sound to be learned.
This dissertation seeks to resolve the apparent conflict between the findings of these two research paradigms through incorporating attentional sets.
Attention can be oriented towards comprehension of the speaker's intended meaning or towards perception of a speaker's pronunciation.
Attention is proposed to affect perceptual learning as follows.
When attention is oriented towards comprehension, more abstract and less context-dependent representations are updated and the perceiver shows a generalized perceptual learning effect (seen in the speech perception literature).
When attention is oriented towards perception, more finely detailed and more context-dependent representations are updated and the perceiver shows a less generalized perceptual learning effect (seen in the psychophysics literature).
This proposal is supported by three experiments.
The first two implement a standard paradigm for perceptual learning in speech perception.
Promoting a more perception-oriented attentional set causes less generalized perceptual learning.
The final experiment uses a novel paradigm where modified sounds embedded in sentences as the exposure.
Perceptual learning is found only when the modified sound is embedded in words that are not predictable from the sentence.
When modified sounds are in predictable words, no perceptual learning is observed.
To account for this lack of perceptual learning, I hypothesize that sounds in predictable words are less reliable than sounds in words in isolation or unpredictable sentences, resulting in no perceptual learning.
Highly predictable sentences are more intelligible as a whole, but the individual words within them are less intelligible.
In the cases where perceptual learning is present, conditions where comprehension-oriented attentional sets are promoted show larger perceptual learning effects than conditions where perception-oriented attentional sets are promoted.
I argue that attention is a key component to the generalization of perceptual learning to new contexts.


% Consider placing version information if you circulate multiple drafts
%\vfill
%\begin{center}
%\begin{sf}
%\fbox{Revision: \today}
%\end{sf}
%\end{center}
