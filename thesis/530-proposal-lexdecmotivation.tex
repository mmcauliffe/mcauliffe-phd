% !TEX TS-program = pdflatex
% !TEX encoding = UTF-8 Unicode

% This is a simple template for a LaTeX document using the "article" class.
% See "book", "report", "letter" for other types of document.

\documentclass[11pt]{article} % use larger type; default would be 10pt

\usepackage[utf8]{inputenc} % set input encoding (not needed with XeLaTeX)

%%% Examples of Article customizations
% These packages are optional, depending whether you want the features they provide.
% See the LaTeX Companion or other references for full information.

%%% PAGE DIMENSIONS
\usepackage{geometry} % to change the page dimensions
\geometry{letterpaper} % or letterpaper (US) or a5paper or....
 \geometry{margin=0.75in} % for example, change the margins to 2 inches all round
% \geometry{landscape} % set up the page for landscape
%   read geometry.pdf for detailed page layout information

\usepackage{graphicx} % support the \includegraphics command and options

% \usepackage[parfill]{parskip} % Activate to begin paragraphs with an empty line rather than an indent

%%% PACKAGES
\usepackage{booktabs} % for much better looking tables
\usepackage{array} % for better arrays (eg matrices) in maths
\usepackage{paralist} % very flexible & customisable lists (eg. enumerate/itemize, etc.)
\usepackage{verbatim} % adds environment for commenting out blocks of text & for better verbatim
\usepackage{subfig} % make it possible to include more than one captioned figure/table in a single float
\usepackage{natbib}
% These packages are all incorporated in the memoir class to one degree or another...

%%% HEADERS & FOOTERS
\usepackage{fancyhdr} % This should be set AFTER setting up the page geometry
\pagestyle{fancy} % options: empty , plain , fancy
\renewcommand{\headrulewidth}{0pt} % customise the layout...
\lhead{}\chead{}\rhead{}
\lfoot{}\cfoot{\thepage}\rfoot{}

%%% SECTION TITLE APPEARANCE
\usepackage{sectsty}
\allsectionsfont{\sffamily\mdseries\upshape} % (See the fntguide.pdf for font help)
% (This matches ConTeXt defaults)

%%% ToC (table of contents) APPEARANCE
\usepackage[nottoc,notlof,notlot]{tocbibind} % Put the bibliography in the ToC
\usepackage[titles,subfigure]{tocloft} % Alter the style of the Table of Contents
\renewcommand{\cftsecfont}{\rmfamily\mdseries\upshape}
\renewcommand{\cftsecpagefont}{\rmfamily\mdseries\upshape} % No bold!

%%% END Article customizations

%%% The "real" document content comes below...

\title{530 Proposal}
\author{Michael McAuliffe}
\date{} % Activate to display a given date or no date (if empty),
         % otherwise the current date is printed 

\begin{document}
%\maketitle
\begin{center}
530 Proposal

Michael McAuliffe
\end{center}

Much research has investigated perceptual learning, the process whereby listeners exposed and adapt to a novel speech pattern from an unfamiliar speaker.  A listener's perceptual system shifts rapidly; for instance, vowel perception can shift solely on the basis of the vowels in a  preceding sentence \citep{LadefogedBroadbent1957}.  Boundaries between fricative categories can be changed by exposure to relatively small amount of ambiguous fricative tokens, but they must be embedded within and associated with a lexical item in a language, and hearing ambiguous productions between two sound categories in nonwords does not induce perceptual learning \citep{NorrisEtAl2003}. However, besides the categorical distinction separating words from nonwords, little is known about the factors that influence linking an shifted or ambiguous production of a sound with a category through a lexical item. My dissertation investigates the factors which can contribute to associating ambiguous atypical productions with a specific sound category through lexical items. 

Given a continuum from a nonword like \emph{dask} to word like \emph{task} that differs only in one sound, listeners in general are more likely to interpret any step along the continuum as the word endpoint rather than the nonword endpoint \citep{Ganong1980}. This lexical bias, also know as the Ganong effect, is exploited in perceptual learning studies to allow for noncanonical, ambiguous productions of a sound to be linked to pre-existing sound categories.

Lexical bias, however, has been shown to be gradient.  When an ambiguous sound is embedded later in a word, listeners are more likely to treat the production as a word than if the same sound is embedded earlier in the word \citep{PittSzostak2012}.  The same study found that when listeners are alerted to the ambiguous sound's presence, and told to listen carefully, lexical biases diminish overall, with listeners less likely to treat the production as a word.    Most studies looking at boundaries between fricative categories use longer (2-3 syllable) lexical items ending with a fricative \citep{NorrisEtAl2003,}, but perceptual learning has also been found following exposure to shorter (1 syllable) lexical items beginning with an ambiguous sound \citep{Clare2014}.  However, specific comparisons in perceptual learning effects across word position in these studies is impossible due to differences in languages under investigation (Dutch or English) and the nature of contrast (/t/-/d/, /s/-/sh/, or /s/-/f/).  The first two experiments in my dissertation address this gap in the literature and investigate the mechanism underlying perceptual learning.

Given the difference in word response rates depending on position in the word, we would predict that listeners exposed to ambiguous sounds earlier in words would be less likely to accept these productions as words as compared to listeners exposed to ambiguous sounds later in words.  In addition, given the reliance of perceptual learning on lexical scaffolding, this lower acceptance rate for the former group would lead to a smaller perceptual learning effect as compared to the latter group.

In these two experiments, listeners will be exposed to ambiguous productions of words containing a single instance of /s/, where the /s/ has been modified to sound more like /sh/ in a lexical decision task.  In one group, the S-Initial group, the critical words will have an /s/ in the onset of the first syllable, like in \emph{cement}, with no /sh/ neighbour, like \emph{shement}.  In the other group, the S-Final group, the critical words will have an /s/ in the onset of the final syllable, like in \emph{tassel}, with no /sh/ neighbour like \emph{tashel}.  In addition, half of each group will be given instructions that the speaker has a modified /s/ and to listen carefully, following \citet{PittSzostak2012}.

The two experiments differ in the salience of the ambiguous stimuli, with salience here defined as the distance from a normal, natural production.  A pretest was conducted with 20 listeners to determine the percentage /s/ response at each step of the critical continua (for instance, from \emph{tassel} to \emph{tashel}).  In the first experiment, the continua steps selected for the critical item  were around 30\% /s/ response, following \citet{ReinischWeberMitterer2013}, which was done to control for any lexical bias.  In the second experiment, the continua steps selected for critical items were around 50\% /s/ response, following other methodologies \citep{NorrisEtAl2003,KraljicSamuel2005}, which should be more susceptible to lexical biases.  The differences in perceptual learning between the groups are predicted be less in the first experiment than in second, as the lessened salience of the stimuli in the second experiment is predicted to allow for a greater role of top-down attention on the processing of the exposure items.

\end{document}
