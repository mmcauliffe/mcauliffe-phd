%% The following is a directive for TeXShop to indicate the main file
%%!TEX root = diss.tex

\chapter{Discussion and conclusions}
\label{chap:conclusion}

This dissertation set out to examine the influence of listener attentional sets on perceptual learning.
Perceptual learning is phenomenon common to many fields involved in cognitive science.
How perceptual learning manifests, however, is quite different.
Perceptual learning in psychophysics is the process of a perceiver aligning their senses to the world.
Perceptual learning in speech perception is process by which perceivers align their perceptual system to an interlocutor to facilitate understanding.
That final part is crucial.
In speech perception, truly accurate perception is immaterial if a listener can understand enough to interact fluently with their interlocutor.

I argue that the perceptual learning as a mechanism is shared between linguistic and non-linguistic domains.
The goals of the tasks and the attentional sets promoted by them are where differences in patterns of behaviour arise.
Perception-oriented tasks in all domains leads to less generalized learning.
Comprehension-oriented tasks lead to more generalized learning.
The first two experiments of this dissertation implement a standard lexically-guided perceptual learning paradigm, but with manipulations to promote perception-oriented attentional sets.
Even in a comprehension-oriented task, promoting more perception-oriented attentional sets leads to less generalized learning.
These results provide a crucial link between fully comprehension-oriented perceptual learning in the lexically-guided paradigm and fully perception-oriented perceptual learning in visually-guided paradigms.
The remainder of this chapter first summarizes the results of the dissertation as they relate to specificity and generalization in the perceptual learning literature.
The four manipulation to promote attentional sets are then examined, followed finally by implications for models of the brain and psycholinguistics.

\section{Specificity and generalization in perceptual learning}

The results of this dissertation speak to the dichotomy between specificity and generalization found in the perceptual learning literature. 
In Experiment 1, participants had larger perceptual learning when they were exposed to ambiguous sounds later in the words rather than at the beginning of words. 
And yet, the testing continua consisted of stimuli with the sibilant at the beginnings of words, which are more similar to the exposure tokens beginning with the ambiguous sound.
Exposure-specificity at the level of sound in context (i.e. word-initial /s/) was tested, but no such specificity was found.
Exposure to /s/ in word-medial position resulted in perceptual learning effects at least as large as for exposure to word-initial /s/.
Perceptual learning occurred at a level of abstraction that is usually not assumed in perceptual learning studies.
Most lexically-guided perceptual learning studies attempt to make the exposure tokens and the categorization similar -- and in some cases, the same -- in order to maximize exposure-specificity effects.
The results of this dissertation show that not only do listeners show perceptual learning effects from stimuli with large degrees of coarticulation (i.e., in the middle of the word) to stimuli without as much coarticulation, in some cases, perceptual learning is greatest in precisely the cases where acoustic distance is greatest.
One aspect that was not tested in the current studies is exposure-specificity at the level of the item.
Perhaps a more perception-oriented attentional set would show greater perceptual learning on the specific exposure items.

The effect of attentional set manipulations in Experiments 1 and 2 suggest that when listeners adopt more perceptually-oriented attentional sets, even within tasks that are oriented toward comprehension, generalization of perceptual learning to new forms is inhibited.
While visually-guided perceptual learning shows comparable effects to lexically-guided perceptual learning \citep{vanLinden2007}, lexically-guided perceptual learning is more likely to be expanded to new contexts  than visually-guided perceptual learning \citetext{\citealp{Norris2003, Kraljic2008a,Reinisch2014}, but see \citealp{Mitterer2013}}.
Visually-guided and psychophysical perceptual learning paradigms often have highly repetitive stimuli with little variation.  
Both of these aspects add to the monotony of the task and the likelihood of perception-oriented attentional sets \citep{Cutler1987}.

Lexically-guided perceptual learning, on the other hand, requires very few instances to affect the perceptual system.
The standard number is around 20 ambiguous tokens within 200 trials, but as few as 10 ambiguous tokens have been shown to have comparable effects \citep{Kraljic2008}.
This dissertation argues for a mechanism of attention that limits updating of expectations to the level at which attention is directed.
If attention is oriented towards perception, then expectations must be resolved at a low level of sensory representation.
At this level, fine details of the sensory input are encoded, and generalization to other stimuli is more difficult due to the fine-grained encoding.
If attention is oriented toward comprehension, expectations can be resolved at a higher level with more abstract sensory representations.
At this level, fine details are not encoded, allowing generalization to other stimuli more readily.
A consequence of this mechanism nicely captures the differences in numbers of stimuli across comprehension-oriented and perception-oriented tasks.
Tokens heard under comprehension-oriented attentional sets should have a relatively large effect on the perceptual system as compared to tokens heard under a perception-oriented attentional set.
A single token updating a more abstract sensory representation will generalize more than many repetitions updating fine-grained sensory representations.
From this, we could predict that word endorsement rate and category boundary shift would be the less (linearly) correlated the more comprehension-oriented participants are.
This prediction is borne out by the lack of correlation between word endorsement rate and cross-over point in Experiment 1 in the Word-final/No Attention condition.
This condition is predicted to have the most comprehension-oriented attentional set of the conditions, and here is the only instance in Experiment 1 where a strong correlation between word endorsement rate and cross-over point is not present.
Participants in this condition have relatively high cross-over points that do not depend as much on the sheer number of tokens endorsed.

\section{Effect of increased linguistic expectations}

The conditions of Experiment 1 that are most similar to previous lexically-guided perceptual learning paradigms are those with no explicit instructions about the /s/ category.
In these conditions, increasing linguistic expectations through lexical bias resulted in larger perceptual learning effects.
I argue that the increased perceptual learning is due to increased maintainance of comprehension-oriented attentional sets by participants in the Word-medial condition.
The participants exposed to a modified /s/ category at the beginnings of words would be more likely to have their attention drawn to the atypicality of the modified /s/ category.
There are two potential scenarios for how this would have affected these participants.
In the first, normal word processing would proceed with the perception of the modified /s/ as part of comprehending the word, but the attentional set would not changed.
In the second, processing the word would trigger an attentional set change that would get reinforced for each new modified /s/ encountered.
The experiments in this dissertation do not definitively answer which scenario is more likely, and it could be that different participants fall into different scenarios.
However, when participants were told about the ambiguity of the /s/, they do not behave any differently if the /s/ is word-initial or word-medial.
This similarity of behavioural patterning suggests the second scenario is more likely, and more perception-oriented attentional sets were adopted as a result of exposure to words beginning with a modified /s/ category.

Increasing linguistic expectations through semantic predictability did not increase perceptual learning.
Semantic predictability has previously been shown to affect perception-oriented tasks a similar way that lexical bias affects them\citep{Connine1987,Borsky1998}
In Experiment 3, participants exposed to the modified /s/ category in high predictability sentences showed no perceptual learning effects at all.
While the Isolation condition (Word-medial condition in Experiment 1) was not significantly different from the Unpredictive condition of Experiment 3, there was a trend toward reduced perceptual learning when the modified sound category was embedded in a sentential context in general.
The lack of a perceptual learning effect from high predictability exposure sentences is reminiscent of studies that find no perceptual learning when a modified /s/ category is embedded in a /str/ cluster that conditions that variation \citep{Kraljic2008a}.
However, there is a difference between the consonant cluster context and the semantic predictability context.
In the consonant cluster, there is a straightforward coarticulatory reason for /s/ to surface as more /\textesh/-like in /str/ clusters, with the /s/ produced more in a postalveolar position due to the upcoming /r/.
For semantic predictability, there is no particular reason why a /s/ should surface more /\textesh/ like in high predictability sentences.
If high semantic predictability can be the attributed cause of /s/ surfacing as more /\textesh/-like, it seems reasonable that there would be a simple relaxing of all auditory categories.

Would semantic predictability have the same effect on nonnative speakers?
Several studies have shown perceptual learning benefits for multiple talkers of an accent using sentential stimuli \citep[and others]{Bradlow2008}.  
Clearly, perceptual learning of accents is possible through hearing sentences of varying predictability, but the phonetic variability involved in those tasks reaches far beyond that involved here.  
The speaker producing the sentences in this dissertation is a native English speaker.
Even with the synthesis applied to the sound files, he is more intelligible than the speakers in studies involving nonnative accents.
The ease of comprehension of the speaker in this study might actually inhibit perceptual learning in sentences, because listeners can leverage so much of their perceptual experience with other speakers of the local dialect.


\section{Attentional control of perceptual learning}

The findings of Experiment 1 support the hypothesis that comprehension-oriented attentional sets produce larger perceptual learning effects than perception-oriented attentional sets.  
Although all participants showed perceptual learning effects, those exposed to the ambiguous sound with increased lexical bias only showed larger perceptual learning effects when the instructions about the speaker's ambiguous sound were withheld.  
Attention on the ambiguous sound equalized the perceptual learning effects across lexical bias.
However, in Experiment 2, there is no such effect of attention.
This suggests that ambiguous sounds farther away from the canonical production induce a more perception-oriented attentional set regardless of explicit instructions.

One question raised by the current results is whether perception-oriented attentional sets always result in decreased perceptual learning.  
The instructions used to focus the listener's attentional set framed the ambiguity in a negative way, with listeners being cautioned to listen carefully to ensure they made the correct decision.  
If the attention were directed to the ambiguous sound by framing the ambiguity in a positive way, would we still see the same pattern of results?
The current mechanism would predict that attention of any kind to signal properties would block the propagation of errors, reducing perceptual learning.
This prediction will be tested in future work.

Attention's role in perceptual learning may extend to the realm of sociolinguistics.  
In sociolinguistics, there are three categories of linguistic variables: indicators, markers, and stereotypes \citep{Labov1972}.
Of these, stereotypes are the most known to speakers of the dialect and speakers of other dialects.
If attention to perception inhibits perceptual learning, then perceptual learning to these stereotype linguistic variables would be inhibited relative to other variables.
Given the scale from indicators to markers to stereotypes is ordered in terms of speaker (or listener) awareness, the role of attention proposed in this dissertation would predict progressively less perceptual learning as awareness increases.
Salient social variants (i.e. r-lessness) have also been found to not be encoded as robustly as canonical productions \citep{Sumner2009}.
Would less salient social variants be learned easier?

\section{Category atypicality}

In Experiment 2, there was no effect of explicit instructions or lexical bias on perceptual learning, with a stable perceptual learning effect present for all listeners.
There are two potential, non-exclusive explanations for the lack of effects.
As stated above, the increased distance to the canonical production drew the listener's attention to the ambiguous productions, resulting in a perception-oriented attentional set.
The second potential explanation is that the productions farther from canonical produce a weaker effect on the updating of a listener's categories, as predicted from the neo-generative model in \citep{Pierrehumbert2002}.
This explanation is supported in part by the weaker correlation between word endorsement rate and cross-over point found in Experiment 2, and the findings of \citet{Sumner2011} where the highest rates of perceptual learning were found when the categories began more typical and gradually became less typical over the course of exposure.
This explanation could be tested straightforwardly by implementing the same gradual shift paradigm used in \citet{Sumner2011} with the manipulations used in this dissertation.

An interesting extension to the current findings would be to observe the perceptual learning effects in a cognitive load paradigm.  
Speech perception under cognitive load has been shown to have greater reliance on lexical information due to weaker initial encoding of the signal \citep{Mattys2011}.  
Following exposure to a modified ambiguous category, we might expect to see less perceptual learning if the exposure was accompanied by high cognitive load.  
\citet{Scharenborg2014}, however, found that hearing loss of older participants did not significantly influence their perceptual learning.  
Therefore it may be that perceptual learning would not fluctuate across cognitive loads.
Higher cognitive loads, however, might allow for more atypical ambiguous stimuli to be learned, due to the increased reliance on lexical information during initial encoding.

It is important to bear in mind that what is typical in one context is not necessarily typical in another.  
The methodology employed for Experiment 3 assumed that expected variation for the category /s/ would be common across all experiments.  
However, it may be that the perfectly ambiguous /s/ category in Experiment 3 was within the range of variation in high predictability sentences. 
In this case, had the category atypicality been more like that of Experiment 2, we may have actually seen more of an effect, perhaps back to the level of Experiment 1.

\section{Implications for cognitive models}

The primary model that this dissertation adopts is that of the predictive coding framework \citep{Clark2013}.
In this model, expectations about sensory input are fed from higher levels of representation to lower ones.
The mismatch between actual sensory input and the expectations is then propagated back to the higher levels as an error signal.
Future expectations are modified based on the error signal.
This framework captures well the basics of perceptual learning, and a similar computational framework has been used to model visually-guided perceptual learning tasks \citep{Kleinschmidt2011}.
However, the attentional mechanism in the predictive coding framework does not work well for some instances of visual attention \citep{Block2013}, or for the current results.
I propose a new attentional mechanism for predictive coding, one in which attention inhibits error propagation beyond the level to which attention is directed, schematized in Figures~\ref{fig:predictivecodingperception} and~\ref{fig:predictivecodingcomprehension}.
Such a mechanism explains both the previous findings and the current results.

Implications extend to other models as well, namely exemplar theoretic frameworks and models of speech processing.
Exemplar theoretic frameworks 

\section{Conclusion}

This dissertation investigated the influences of attention and linguistic salience on perceptual learning in speech perception.
Perceptual learning was modulated by the attentional set of the listener.
Increasing linguistic expectations through increased lexical bias induced more comprehension-oriented attentional sets, resulting in larger perceptual learning effects.
Increasing semantic predictability resulted in no perceptual learning effects, likely due to the attribution of the modified sound category to reduced speech clarity.
Inducing perception-oriented attentional sets through explicit instructions or making the modified sound category more atypical resulted in small, but robust perceptual learning effects.
These results support a greater role of attention than previously assumed in predictive coding frameworks, such as the proposed propagation-blocking mechanism.
Finally, these results suggest that the degree to which listeners perceptually adapt to a new speaker is under their control to the same degree as attentional set adoption.
Given the robust perceptual learning effects across all conditions, perceptual learning is a largely automatic process.


