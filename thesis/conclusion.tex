%% The following is a directive for TeXShop to indicate the main file
%%!TEX root = diss.tex

\chapter{Conclusions}
\label{chap:conclusion}

\section{Effect of increased linguistic expectations}

Bias for a word was manipulated in two ways, through position of the ambiguous sound in the word and through sentential manipulations.
The conditions that correspond most closely to those in other lexically-guided perceptual learning experiments \citep{Norris2003, Kraljic2005} are those in the No Attention condition of Experiment 1.
Under those conditions, there is a clear effect of lexical bias, such that listeners exposed to ambiguous stimuli with greater lexical bias update their speaker-specific category for that sound more.


\section{Attentional control of perceptual learning}

The findings of Experiment 1 suggest also that perceptual learning is not a wholly automatic process.  Although all participants showed perceptual learning effects, those exposed to the ambiguous sound with increased lexical bias only showed larger perceptual learning effects when the instructions about the speaker's ambiguous sound were witheld.  Attention on the ambiguous sound equalized the perceptual learning effects across lexical bias.
However, in Experiment 2, there is no such effect of attention, suggesting that the ambiguous sounds that were farther away from the canonical production drew the listener's attention to those sounds.

One question raised by the current results is whether attention always decreases perceptual learning.  
The instructions used to focus the listener's attentional set framed the ambiguity in a negative way, with listeners being cautioned to listen carefully to ensure they made the correct decision.  
If the attention were directed to the ambiguous sound by framing the ambiguity in a positive way, would we still see the same pattern of results?  
I would predict that there would be a greater tendency to endorse ambiguous productions as words when the attention is positive as compared to both no attention and negative attention, and would therefore lead to greater perceptual learning effects.  
This prediction will be tested in future work.

\section{Specificity versus generalization in perceptual learning}

The results of this dissertation speak to dichotomy between specificity and generalization found in perceptual learning work. 
In Experiment 1, greater perceptual learning was shown by participants exposed to ambiguous sounds later in the words, not to those at the beginnings of words. 
And yet, the testing continua consisted of stimuli with the sibilant at the beginnings of words, which are more similar to the words beginning with the ambiguous sound.

The effect of attention in Experiments 1 and 2 suggest that when speakers are focused on the signal, or have their attention drawn to the signal, they are less likely to generalize across forms.
When attention on the signal is not present, the available lexical connections can assist in greater generalization to new forms.
These different modes of perceptual learning may account for some differences that have been observed in the literature.
While visually-guided perceptual learning shows comparable effects to lexically-guided perceptual learning \citep{vanLinden2007}, lexically-guided perceptual learning is more likely to be expanded to new contexts \citep{Norris2003, Kraljic2008a}, though with some restrictions \citep{Mitterer2013}, than visually-guided perceptual learning \citep{Reinisch2014}.
%The methodologies used to induces perceptual learning in these two ways differ in ways that might confound this generalization.
%Lexically-guided perceptual learning uses relatively few ambiguous tokens, but ones embedded in an array of lexical contexts whereas visually-guided perceptual learning uses many more ambiguous token instances, but embedded in very few frames.

\section{Distance to canonical production}

In Experiment 2, there was no effect of attention or lexical bias on perceptual learning, with a stable effect present for all listeners.
There are two potential, non-exclusive explanations for the lack of effects.
As stated above, the increased distance to the canonical production drew the listener's attention to the ambiguous productions, resulting in a similar pattern as for listeners in Experiment 1 in the Attention condition.
The second poential explanation is that the productions farther from canonical produce a weaker effect on the updating of a listener's categories, as predicted from the neo-generative model in \citep{Pierrehumbert2002}.
This explanation is supported in part by the weaker correlation between word endorsement rate and crossover point found in Experiment 2, and the findings of \citet{Sumner2011} where the most perceptual learning was found when the categories begin like the listener expects and gradually shift toward the speaker's actual boundaries over the course of exposure.

These two explanations could be tested if the predictions for positive attention are borne out.
If positive attention does increase word endorsement rate as predicted, it should have an effect on exposure items farther away from canonical productions as well.
If those productions do have a weaker effect on updating listener categories, then we should see a weaker perceptual learning effect across ambiguous stimuli types despite consistent word endorsement rates.

An additional way to test the second explanation would be to implement a paradigm similar to that used in \citet{Sumner2011}.  
It is unlikely that the non-nativeness of the speaker is the primary reason for the findings.  
While a listener might expect a non-native speaker to produce speech farther from the expected distributions of native speakers, the gradual transition from English-like stops to French-like stops is at least as unrealistic as the other conditions in \citet{Sumner2011}. 
One study that included a manipulation to the order of presentation of ambiguous and unambiguous stimuli is \citet{Kraljic2008}, which found that when speakers are presented with 10 ambiguous /s/-/\textesh/ stimuli and 10 nonambiguous /s/ stimuli in different orders, perceptual learning differs.  
When the ambiguous stimuli are presented first followed by the nonambiguous ones, listeners show a perceptual learning effect, but no perceptual learning effect is found when the order is reversed.  
They argue that listeners only adapt their categories if the ambiguous stimuli are reliable evidence.  
So in the case where participants heard the ambiguous stimuli after the nonambiguous ones, the ambiguous stimuli would not be interpreted as reliable in light of the previous tokens that the listener had heard.  
Because the difference between the initial stimuli and the later ones in \citet{Kraljic2008} was a categorical one, it may have drawn the attention of the listener like the stimuli used in Experiment 2. 
Perhaps a more the gradient change as in \citet{Sumner2011} could keep the differences beneath the attentional threshold of the listener and thus induce greater perceptual learning.
If participants had a larger perceptual learning effect,  as compared to the participants in Experiment 1, after exposure to a presentation order that began as more canonical productions and then gradually shifted to productions for the 50\% ambiguous point, then the predictions of \citet{Pierrehumbert2002} would be further supported.

This disseratation investigated the interaction between linguistic, attentional and signal factors influencing perceptual learning in speech perception.
Increased linguistic bias resulted in larger perceptual learning effects.
However, perceptual learning was modulated by the attention of the listener and the propterties of the signal.
If a listener was warned of the ambiguous sound, the effect of increased linguistic bias on perceptual learning disappeared.
Likewise, if the ambiguous sound was farther from the intended target, there was no effect of linguistic bias or attention on perceptual learning.
These results suggest that the degree to which listeners perceptually adapt to a new speaker is under their control to a degree.
All participants as a whole exhibited a perceptual learning effect, indicating that perceptual learning is to some extent an automatic process.


