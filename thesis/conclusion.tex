%% The following is a directive for TeXShop to indicate the main file
%%!TEX root = diss.tex

\chapter{Conclusions}


\section{Specificity versus generalization}

In Experiment 1, greater perceptual learning was shown by participants exposed to ambiguous sounds later in the words, not to those at the beginnings of words

And yet, the testing continua consisted of stimuli with the sibilant at the beginnings of words, which are more similar to the S-Initial words than the S-Final words

Attention removed this effect of position, with the perceptual learning effect identical across participants in the exposure condition when they were told to pay attention to the sibilant

\section{Distance to canonical}

In Experiment 2, there was no effect of attention or lexical bias on perceptual learning, with a stable effect present for all listeners

There are two potential explanations for the lack of effects:

One, the increased distance to the canonical production increased the salience of the production and drew the listener's attention to the ambiguous productions, resulting in a similar pattern for listeners in Experiment 1 in the Attention condition

Two, the productions farther from canonical produce a weaker effect on the updating of a listener's categories

\begin{itemize}
\item Predicted by a Pierrehumbert model

\item This is supported somewhat by the weaker correlation between word endorsement rate and crossover point found in Experiment 2
	
\item Also supported by the findings of \citet{Sumner2011} where the most perceptual learning was found when the categories begin like the listener expects and gradually shift toward the speaker's actual boundaries over the course of exposure
\end{itemize}

\section{Effect of increased bias}

Bias for a word was manipulated in two ways, through position of the ambiguous sound in the word and through sentential manipulations

