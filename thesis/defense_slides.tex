% !TEX TS-program = pdflatex
% !TEX encoding = UTF-8 Unicode

% This file is a template using the "beamer" package to create slides for a talk or presentation
% - Giving a talk on some subject.
% - The talk is between 15min and 45min long.
% - Style is ornate.

% MODIFIED by Jonathan Kew, 2008-07-06
% The header comments and encoding in this file were modified for inclusion with TeXworks.
% The content is otherwise unchanged from the original distributed with the beamer package.

\documentclass{beamer}


% Copyright 2004 by Till Tantau <tantau@users.sourceforge.net>.
%
% In principle, this file can be redistributed and/or modified under
% the terms of the GNU Public License, version 2.
%
% However, this file is supposed to be a template to be modified
% for your own needs. For this reason, if you use this file as a
% template and not specifically distribute it as part of a another
% package/program, I grant the extra permission to freely copy and
% modify this file as you see fit and even to delete this copyright
% notice. 

\usepackage{multimedia}
\usepackage{tipa}
\usepackage{natbib}
    \renewcommand{\bibsection}{\subsubsection*{\bibname } }
\usepackage{tikz}
\usetikzlibrary{arrows,shapes}

\newcommand{\tikzmark}[1]{\tikz[remember picture] \node[coordinate] (#1) {#1};}
\newcommand\mytextbullet{\leavevmode%
\usebeamertemplate{itemize item}}
\newcommand{\btVFill}{\vskip0pt plus 1filll}

\mode<presentation>
{
  \usetheme{Warsaw}
  % or ...

  \setbeamercovered{transparent}
  % or whatever (possibly just delete it)
}


\usepackage[english]{babel}
% or whatever

\usepackage[utf8]{inputenc}
% or whatever

\usepackage{times}
\usepackage[T1]{fontenc}
% Or whatever. Note that the encoding and the font should match. If T1
% does not look nice, try deleting the line with the fontenc.


\title
{Attention and salience in lexically-guided perceptual learning}

\author
{Michael McAuliffe}
% - Use the \inst{?} command only if the authors have different
%   affiliation.

\institute
{}
% - Use the \inst command only if there are several affiliations.
% - Keep it simple, no one is interested in your street address.

\date
{PhD Defense}

\subject{Talks}
% This is only inserted into the PDF information catalog. Can be left
% out. 



% If you have a file called "university-logo-filename.xxx", where xxx
% is a graphic format that can be processed by latex or pdflatex,
% resp., then you can add a logo as follows:

% \pgfdeclareimage[height=0.5cm]{university-logo}{university-logo-filename}
% \logo{\pgfuseimage{university-logo}}





% If you wish to uncover everything in a step-wise fashion, uncomment
% the following command: 

%\beamerdefaultoverlayspecification{<+->}


\begin{document}

\begin{frame}
  \titlepage
\end{frame}

\begin{frame}{Outline}
\tableofcontents
\end{frame}

\section{Background}

\subsection{Sources of variation}

\begin{frame}{Sources of variation}

Example: /s/
\vfill
\begin{minipage}[t]{0.45\textwidth}
Speaker

\begin{itemize}
\item Indexical
\begin{itemize}
\item Accent
\item Gender
\end{itemize}
\item Contextual
\begin{itemize}
\item Style
\item Speaking rate
\item Coarticulation (/st\textturnr/)
\item Position in word
\item Predictability
\end{itemize}

\end{itemize}
\end{minipage}
\hfill
\begin{minipage}[t]{0.45\textwidth}
Listener

\begin{itemize}
\item Indexical
\begin{itemize}
\item Accent
\item Perceived accent
\item Perceived gender
\end{itemize}
\item Contextual
\begin{itemize}
\item Speaking rate
\item Coarticulation (/st\textturnr/)
\end{itemize}
\item Attention
\end{itemize}
\end{minipage}
\btVFill
\begin{flushright}
\scriptsize
\citet{Strand1996, Kraljic2005,Clopper2008, Pitt2012}
\end{flushright}
\end{frame}

\begin{frame}{Perceptual constancy}
\vfill
\begin{center}
Despite variation, listeners can interpret variable productions as a single word type
\vfill
\movie{\includegraphics[width=0.1\textwidth]{pictures/sock}}{audio/variation.wav}

\end{center}
\btVFill

\begin{flushright}
\scriptsize
\citet{Shankweiler1977, Kuhl1979,Sumner2013}
\end{flushright}
\end{frame}

\begin{frame}{Perceptual learning}
\movie{\includegraphics[width=0.45\textwidth]{pictures/exposure}}{audio/exposure.wav}
\hfill
\movie{\includegraphics[width=0.45\textwidth]{pictures/categorization}}{audio/categorization.wav}

\vfill
\textbf{Research question:}

How do changes to a listener's attentional set in exposure affect perceptual learning in future categorization?
\vfill

\end{frame}

\begin{frame}{Sources of variation}

Example: /s/
\vfill
\begin{minipage}[t]{0.45\textwidth}
Speaker

\begin{itemize}
\item Indexical
\begin{itemize}
\item Accent
\item Gender
\end{itemize}
\item Contextual
\begin{itemize}
\item Style
\item Speaking rate
\item Coarticulation (/st\textturnr/)
\item \textbf{Position in word} \tikzmark{n1} 
\item \textbf{Predictability} \tikzmark{n2} 
\end{itemize}

\end{itemize}
\end{minipage}
\hfill
\begin{minipage}[t]{0.45\textwidth}
Listener

\begin{itemize}
\item Indexical
\begin{itemize}
\item Accent
\item Perceived accent
\item Perceived gender
\end{itemize}
\item Contextual
\begin{itemize}
\item Speaking rate
\item Coarticulation (/st\textturnr/)
\end{itemize}
\item[\tikzmark{t1} \mytextbullet] \textbf{Attention}
\end{itemize}

\end{minipage}

\begin{tikzpicture}[remember picture,overlay]
        %\path[draw=magenta,thick,->]<3-> ([yshift=3mm]n1) to ++(0,3mm) to [out=0, in=0,distance=2.5in] (t1);
   \path[draw=magenta,thick,->] ([yshift=1.5mm]n1) -- ([yshift=1.5mm]t1);
\end{tikzpicture}
\begin{tikzpicture}[remember picture,overlay]
        %\path[draw=magenta,thick,->]<3-> ([yshift=3mm]n1) to ++(0,3mm) to [out=0, in=0,distance=2.5in] (t1);
   \path[draw=magenta,thick,->] ([yshift=1.5mm]n2) -- ([yshift=1.5mm]t1);
\end{tikzpicture}
\btVFill
\begin{flushright}
\scriptsize
\citet{Strand1996, Kraljic2005,Clopper2008, Pitt2012}
\end{flushright}
\end{frame}

\subsection{Attentional sets}

\begin{frame}{Attentional sets}

\textbf{Perception-oriented}
\begin{itemize}
\item Focus on perceiving a specific pronunciation
\item Tasks
\begin{itemize}
\item Phoneme-monitoring
\item Discrimination of non-speech stimuli
\end{itemize}
\item Perceptual learning in psychophysics
\begin{itemize}
\item Perceptual learning effects don't generalize to new stimuli
\end{itemize}
\item Visually-guided perceptual learning
\begin{itemize}
\item Mixed results in generalizing to new stimuli
\end{itemize}
\item Promoted by experimental design
\begin{itemize}
\item Repetitive stimuli
\item More nonwords than words
\end{itemize}
\end{itemize}

\btVFill
\begin{flushright}
\scriptsize
\citet{Cutler1987, Ahissar1993, Pitt2012, Reinisch2014}
\end{flushright}


\end{frame}

\begin{frame}{Attentional sets}

\textbf{Comprehension-oriented}
\begin{itemize}
\item Focus on comprehending meaning
\item Tasks
\begin{itemize}
\item Word recognition
\item Word identification
\end{itemize}
\item Perceptual learning in speech perception
\begin{itemize}
\item Generalization to new items (and sometimes new speakers)
\end{itemize}
\item Hypothesized to be similar to normal language use
\end{itemize}

\btVFill
\textbf{Hypothesis:}

Comprehension-oriented attentional sets allow for greater generalization than perception-oriented attentional sets.
\btVFill
\begin{flushright}
\scriptsize
\citet{Norris2003, Kraljic2005, Pitt2012, Reinisch2013}
\end{flushright}

\end{frame}

\begin{frame}{Attentional set manipulation}
\btVFill
Explicit instructions
\begin{itemize}
\item ``This speaker's `s' sounds are ambiguous''
\item Promote perception-oriented attentional set
\end{itemize}

\btVFill
\begin{flushright}
\scriptsize
\citet{Pitt2012}
\end{flushright}
\end{frame}

\begin{frame}{Attentional set manipulation}

Perceptual salience of modified /s/
\begin{itemize}
\item The less predictable an element, the higher its salience
\item Increase the likelihood of listener noticing modification
\item Promote perception-oriented attentional set
\item Assumption: similar to increasing the number of /s/ trials relative to filler trials
\end{itemize}

Position in word
\begin{itemize}
\item Listeners are more tolerant of variation later in the word
\item Word-initial modified /s/ should be more salient 
\end{itemize}

Category typicality
\begin{itemize}
\item Productions that are unexpected for a category are more likely to be noticed (salient)
\end{itemize}

\btVFill
\begin{flushright}
\scriptsize
\citet{Pitt2012}
\end{flushright}
\end{frame}

\section{Experiments 1 and 2}

\subsection{Set up}

\begin{frame}{Experiments 1 and 2}

Experiment 1
\begin{itemize}
\item Lexical decision exposure task
\item Across subject factors
\begin{itemize}
\item Instructions
\item Position of modified /s/ in words (\movie{Word-initial}{audio/submarine_exp1.wav} vs \movie{word-medial}{audio/whistle_exp1.wav})
\end{itemize}
\item 50\% word response rate in a pre-test
\end{itemize}
Experiment 2
\begin{itemize}
\item Same design and materials as Experiment 1
\item 30\% word reponse rate in the pre-test (more atypical /s/; \movie{Word-initial}{audio/submarine_exp2.wav} vs \movie{word-medial}{audio/whistle_exp2.wav})
\end{itemize}

\end{frame}

\begin{frame}{Experiment 1 and 2 predictions}

\begin{itemize}
\item Hypothesis 1:
\begin{itemize}
\item Perceptual learning is affected by attentional sets
\item Perceptual learning should be less where perception-oriented attentional sets are promoted
\end{itemize}
\item Hypothesis 2:
\begin{itemize}
\item Perceptual learning is wholly automatic and consistent
\item Equal perceptual learning effects across all conditions
\end{itemize}
\item Hypothesis 3:
\begin{itemize}
\item Perceptual learning effects are dependent on similarity
\item Word-initial exposure > Word-medial exposure
\end{itemize}
\end{itemize}
\end{frame}

\subsection{Results}

\begin{frame}{Experiment 1}

\begin{minipage}{0.45\textwidth}
\textbf{Exposed to ambiguous /s/}
\begin{itemize}
\item 50\% between /s/ and /\textesh/
\end{itemize}

\textbf{Attention}
\begin{itemize}
\item No /s/-oriented instructions
\item Told /s/ would be ambiguous
\end{itemize}

\textbf{Position of /s/}
\begin{itemize}
\item \emph{Word initial}
\item Word medial
\end{itemize}
\end{minipage}
\hfill
\begin{minipage}{0.4\textwidth}
\includegraphics[width=1.0\textwidth]{graphs/exp1_categresults_present2-initial}
\end{minipage}

\end{frame}

\begin{frame}{Experiment 1}

\begin{minipage}{0.45\textwidth}
\textbf{Exposed to ambiguous /s/}
\begin{itemize}
\item 50\% between /s/ and /\textesh/
\end{itemize}

\textbf{Attention}
\begin{itemize}
\item No /s/-oriented instructions
\item Told /s/ would be ambiguous
\end{itemize}

\textbf{Position of /s/}
\begin{itemize}
\item Word initial
\item \emph{Word medial}
\end{itemize}
\end{minipage}
\hfill
\begin{minipage}{0.4\textwidth}
\includegraphics[width=1.0\textwidth]{graphs/exp1_categresults_present2-final}
\end{minipage}

\end{frame}

\begin{frame}{Experiment 2}

\begin{minipage}{0.45\textwidth}
\textbf{Exposed to ambiguous /s/}
\begin{itemize}
\item More like /\textesh/ than /s/
\end{itemize}

\textbf{Attention}
\begin{itemize}
\item No /s/-oriented instructions
\item Told /s/ would be ambiguous
\end{itemize}

\textbf{Position of /s/}
\begin{itemize}
\item \emph{Word initial}
\item Word medial
\end{itemize}
\end{minipage}
\hfill
\begin{minipage}{0.4\textwidth}
\includegraphics[width=1.0\textwidth]{graphs/exp2_categresults_present2-initial}
\end{minipage}

\end{frame}

\begin{frame}{Experiment 2}

\begin{minipage}{0.45\textwidth}
\textbf{Exposed to ambiguous /s/}
\begin{itemize}
\item More like /\textesh/ than /s/
\end{itemize}

\textbf{Attention}
\begin{itemize}
\item No /s/-oriented instructions
\item Told /s/ would be ambiguous
\end{itemize}

\textbf{Position of /s/}
\begin{itemize}
\item Word initial
\item \emph{Word medial}
\end{itemize}
\end{minipage}
\hfill
\begin{minipage}{0.4\textwidth}
\includegraphics[width=1.0\textwidth]{graphs/exp2_categresults_present2-final}
\end{minipage}

\end{frame}

\subsection{Summary}

\begin{frame}{Summary}

\begin{itemize}
\item Results align with attentional sets
\item Fine-grained similarity did not appear to play a role
\begin{itemize}
\item Word-intial exposure <= word-medial exposure
\end{itemize}
\item Conditions promoting a perception-oriented attentional set
\begin{itemize}
\item Still showed perceptual learning
\item Had smaller perceptual learning effects
\item Did not differ from one another
\end{itemize}
\item Task was comprehension-oriented (identifying word)
\item Experiment 3 attempts to further promote comprehension-oriented attentional sets
\end{itemize}

\end{frame}

\section{Experiment 3}

\subsection{Set up}

\begin{frame}{Experiment 3}

\begin{itemize}
\item Novel cross-modal paradigm
\begin{itemize}
\item Auditory sentences
\item Identification of picture corresponding to final word in sentence
\item Same word-medial modified /s/ stimuli
\item Final targets were predictable or unpredictable
\end{itemize}
\item Across subjects
\begin{itemize}
\item Instructions (identical to Experiments 1 and 2)
\item Modified /s/ only in \movie{predictable}{audio/P_whistle.wav} or \movie{unpredictable}{audio/U_whistle.wav} words
\end{itemize}
\item Predictable words are predicted to have lower salience than unpredictable words
\end{itemize}

\end{frame}

\begin{frame}{Experiment 3 predictions}

\begin{itemize}
\item Hypothesis 1:
\begin{itemize}
\item Words in isolation == words at the end of sentences
\item Perceptual learning effects == Experiment 1's Word-medial conditions
\end{itemize}
\item Hypothesis 2:
\begin{itemize}
\item Words in isolation =/= words at the end of sentences
\item Perceptual learning effect < Experiment 1's Word-medial conditions
\end{itemize}
\item Hypothesis 3:
\begin{itemize}
\item High predictability is associated with less distinct acoustics
\item Perceptual learning is not found in coarticulation contexts (/st\textturnr/)
\item No perceptual learning effect in predictable condition
\end{itemize}
\end{itemize}
\btVFill
\begin{flushright}
\scriptsize
\citet{Clopper2008, Scarborough2010,Kraljic2008a}
\end{flushright}
\end{frame}

\subsection{Results}

\begin{frame}{Experiment 3}

\begin{minipage}{0.45\textwidth}
\begin{itemize}
\item \textbf{Exposed to ambiguous /s/}
\begin{itemize}
\item Halfway between /s/ and /\textesh/
\item In sentences
\end{itemize}

\item \textbf{Attention}
\begin{itemize}
\item No /s/-oriented instructions
\item Told /s/ would be ambiguous
\end{itemize}

\item \textbf{Predictability of final /s/ words}
\begin{itemize}
\item \emph{Unpredictable}
\item Predictable
\end{itemize}
\end{itemize}
\end{minipage}
\hfill
\begin{minipage}{0.4\textwidth}
\includegraphics[width=1.0\textwidth]{graphs/exp3_categresults_present2-unpredictable}
\end{minipage}

\end{frame}

\begin{frame}{Experiment 3}

\begin{minipage}{0.45\textwidth}
\begin{itemize}
\item \textbf{Exposed to ambiguous /s/}
\begin{itemize}
\item Halfway between /s/ and /\textesh/
\item In sentences
\end{itemize}

\item \textbf{Attention}
\begin{itemize}
\item No /s/-oriented instructions
\item Told /s/ would be ambiguous
\end{itemize}

\item \textbf{Predictability of final /s/ words}
\begin{itemize}
\item Unpredictable
\item \emph{Predictable}
\end{itemize}
\end{itemize}
\end{minipage}
\hfill
\begin{minipage}{0.4\textwidth}
\includegraphics[width=1.0\textwidth]{graphs/exp3_categresults_present2-predictable}
\end{minipage}

\end{frame}

\begin{frame}{Isolation vs Sentences}

\includegraphics{graphs/exp23_categresults_present}
\end{frame}

\subsection{Summary}

\begin{frame}{Summary}

\begin{itemize}
\item Unpredictable exposure showed a similar pattern to words in isolation
\item Predictable exposure showed no perceptual learning effect
\begin{itemize}
\item Similar to studies using a coarticulation context (/st\textturnr/)
\item Despite consistent durations for words and sibilants across the two sentence types
\end{itemize}
\end{itemize}

\end{frame}

\section{Discussion}

\begin{frame}{Discussion}

\begin{itemize}
\item Attentional sets affected perceptual learning
\begin{itemize}
\item Conditions that did not promote perception-oriented attentional sets showed larger effects
\end{itemize}
\item Predictability was likely not an attentional set manipulation
\begin{itemize}
\item Instead, allowed for attribution of the modified category to predictability
\end{itemize}
\item Implications for theoretical models
\begin{itemize}
\item Supports hierarchical representations
\item Attention to episodic representations or specific pronunciations inhibits learning in abstract categories
\end{itemize}
\end{itemize}


\end{frame}

\begin{frame}[allowframebreaks]{References}%in case more than 1 slide needed
    \footnotesize
\bibliographystyle{apalike}
\bibliography{biblio}
    
\end{frame}

\end{document}


