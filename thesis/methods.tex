%% The following is a directive for TeXShop to indicate the main file
%%!TEX root = diss.tex

\chapter{Methodology}

Stuff I have done



\section{Lexical decision}

\subsection{Materials}

\subsubsection{Exposure}

Participants were exposed to 200 stimulus items in the lexical decision task.  All participants were exposed to the same 100 nonwords and the same 80 filler words, but half the participants were exposed to 20 critical words with /s/ in the onset of the initial syllable, and half were exposed to 20 critical words with /s/ in the onset of the final syllable.  The nonwords contained no instances of sibilants (/s, z, sh, zh, ch, jh/).  For the filler items, 60 tokens had no sibilants, 10 had /sh/ in the onset of the initial syllable, and 10 had /sh/ in the onset of the final syllable.  The critical words only had one instance of /s/ in the word and no other sibilants.  Differences in log frequency from SUBTLEXUS (CITE) and number of syllables across the subgroups of stimuli were not significant (TABLE?). PUT IN APPENDIX


\subsubsection{Categorization}

The items for categorization were four pairs of words (\emph{sack}-\emph{shack}, \emph{sigh}-\emph{shy}, \emph{sin}-\emph{shin}, and \emph{sock}-\emph{shock}) which were monosyllabic and differed only in their initial sibilant.  Two of the pairs had a higher log frequency from SUBTLEXUS for the /s/ word, and two had higher log frequency for the /sh/ word.  \emph{Sack}-\emph{shack} had frequencies of 1.11 lfpm and 0.75 lfpm, \emph{sigh}-\emph{shy} had frequencies of 0.53 lfpm and 1.26 lfpm, \emph{sin}-\emph{shin} had frequencies of 1.20 lfpm and 0.48 lfpm, and \emph{sock}-\emph{shock} had frequencies of 0.95 lfpm and 1.46 lfpm.
	

\subsubsection{Synthesis}

All stimuli items were resynthesized using STRAIGHT (CITE).  For the critical exposure tokens, a normal /s/ version of the word was produced and then a /sh/ version was produced with instructions for consistent prosody across the two tokens.  An 11-step continuum from 100\% /s/ to 100\% /sh/ was synthesized using STRAIGHT and a pretest determined the stimulus to use.

Critical

	Recorded s-version of word and sh-version of word

	Morphed from 100\% s-version to 100\% sh-version over 11 steps

	Pretest to determine optimal ambiguous stimuli

	Follow Reinisch and Mitterer and use the continuum step where S-identification falls below 30\% (to account for Ganong effect)

Filler

	Synthesized using STRAIGHT to provide consistent quality across all stimuli

\subsection{Participants}

Two categorical between subject factors:

	Position of SSH in word (initial syllable, final syllable)
	Attention (No attention, attention)

Power estimated 100 subjects

4 conditions

25 subjects per condition

\subsection{Procedure}

Participants in the experimental conditions completed two tasks, an exposure task and a categorization task.  The exposure task was a lexical decision task, where participants heard auditory stimuli and were instructed to respond with either "word" if they thought what they heard was a word or "nonword" if they didn't think it was a word.  The buttons corresponding to "word" and "nonword" were counterbalanced across participants.

In the categorization task, participants heard an auditory stimulus and had to categorize it as one of two words, differing only in the onset sibilant (s vs sh).  The buttons corresponding to the words were counterbalanced across participants.

Participants were instructed that there would two tasks in the experiment, and both tasks were explained at the beginning to remove experimenter interaction between exposure and categorization.  Additionally, participants in the Attention condition received additional instructions that the speaker's "s" sounds were sometimes ambiguous, and to listen carefully to ensure correct responses in the lexical decision.

