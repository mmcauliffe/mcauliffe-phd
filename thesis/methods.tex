%% The following is a directive for TeXShop to indicate the main file
%%!TEX root = diss.tex

\chapter{Methodology}

Stuff I have done



\section{Lexical decision}

\subsection{Materials}

\subsubsection{Exposure}

200 stimuli

	100 words
	100 nonwords

40 critical words

	s in syllable onset

	20 s-initial words
	20 s-final words

80 filler words

	10 sh-initial words
	10 sh-final words
	60 words with no sibilants

No significant differences in log frequency or number of syllables across the types of words (s-initial, s-final, sh-initial, sh-final, filler)

100 nonwords

	No sibilants
	created from the filler words

\subsubsection{Categorization}

4 word pairs

	sin-shin
	sock-shock
	sack-shack
	sigh-shy

2 pairs with s word more frequent
	sigh (0.53 log count per million)-shy (1.26 log count per million)
	sock (0.95 log count per million)-shock (1.46 log count per million)

2 pairs with sh-word more frequent
	sin (1.20 log count per million)-shin (0.48 log count per million)
	sack (1.11 log count per million)-shack (0.75 log count per million)
	

\subsubsection{Synthesis}

Used STRAIGHT

Critical

	Recorded s-version of word and sh-version of word

	Morphed from 100\% s-version to 100\% sh-version over 11 steps

	Pretest to determine optimal ambiguous stimuli

	Follow Reinisch and Mitterer and use the continuum step where S-identification falls below 30\% (to account for Ganong effect)

Filler

	Synthesized using STRAIGHT to provide consistent quality across all stimuli


\subsection{Participants}

Two categorical between subject factors:

	Position of SSH in word (initial syllable, final syllable)
	Attention (No attention, attention)

Power estimated 100 subjects

4 conditions

25 subjects per condition

\subsection{Procedure}



