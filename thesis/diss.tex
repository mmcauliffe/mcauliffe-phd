%%%%%%%%%%%%%%%%%%%%%%%%%%%%%%%%%%%%%%%%%%%%%%%%%%%%%%%%%%%%%%%%%%%%%%
% Template for a UBC-compliant dissertation
% At the minimum, you will need to change the information found
% after the "Document meta-data"
%
%!TEX TS-program = pdflatex
%!TEX encoding = UTF-8 Unicode

%% The ubcdiss class provides several options:
%%   gpscopy (aka fogscopy)
%%       set parameters to exactly how GPS specifies
%%         * single-sided
%%         * page-numbering starts from title page
%%         * the lists of figures and tables have each entry prefixed
%%           with 'Figure' or 'Table'
%%       This can be tested by `\ifgpscopy ... \else ... \fi'
%%   10pt, 11pt, 12pt
%%       set default font size
%%   oneside, twoside
%%       whether to format for single-sided or double-sided printing
%%   balanced
%%       when double-sided, ensure page content is centred
%%       rather than slightly offset (the default)
%%   singlespacing, onehalfspacing, doublespacing
%%       set default inter-line text spacing; the ubcdiss class
%%       provides \textspacing to revert to this configured spacing
%%   draft
%%       disable more intensive processing, such as including
%%       graphics, etc.
%%

% For submission to GPS
\documentclass[gpscopy,onehalfspacing,11pt]{ubcdiss}

% For your own copies (looks nicer)
% \documentclass[balanced,twoside,11pt]{ubcdiss}

%%%%%%%%%%%%%%%%%%%%%%%%%%%%%%%%%%%%%%%%%%%%%%%%%%%%%%%%%%%%%%%%%%%%%%
%%%%%%%%%%%%%%%%%%%%%%%%%%%%%%%%%%%%%%%%%%%%%%%%%%%%%%%%%%%%%%%%%%%%%%
%%
%% FONTS:
%% 
%% The defaults below configures Times Roman for the serif font,
%% Helvetica for the sans serif font, and Courier for the
%% typewriter-style font.  Configuring fonts can be time
%% consuming; we recommend skipping to END FONTS!
%% 
%% If you're feeling brave, have lots of time, and wish to use one
%% your platform's native fonts, see the commented out bits below for
%% XeTeX/XeLaTeX.  This is not for the faint at heart. 
%% (And shouldn't you be writing? :-)
%%

%% NFSS font specification (New Font Selection Scheme)
\usepackage{times,mathptmx,courier}
\usepackage[scaled=.92]{helvet}

%% Math or theory people may want to include the handy AMS macros
%\usepackage{amssymb}
%\usepackage{amsmath}
%\usepackage{amsfonts}

%% The pifont package provides access to the elements in the dingbat font.   
%% Use \ding{##} for a particular dingbat (see p7 of psnfss2e.pdf)
%%   Useful:
%%     51,52 different forms of a checkmark
%%     54,55,56 different forms of a cross (saltyre)
%%     172-181 are 1-10 in open circle (serif)
%%     182-191 are 1-10 black circle (serif)
%%     192-201 are 1-10 in open circle (sans serif)
%%     202-211 are 1-10 in black circle (sans serif)
%% \begin{dinglist}{##}\item... or dingautolist (which auto-increments)
%% to create a bullet list with the provided character.
\usepackage{pifont}

%%%%%%%%%%%%%%%%%%%%%%%%%%%%%%%%%%%%%%%%%%%%%%%%%%%%%%%%%%%%%%%%%%%%%%
%% Configure fonts for XeTeX / XeLaTeX using the fontspec package.
%% Be sure to check out the fontspec documentation.
%\usepackage{fontspec,xltxtra,xunicode}	% required
%\defaultfontfeatures{Mapping=tex-text}	% recommended
%% Minion Pro and Myriad Pro are shipped with some versions of
%% Adobe Reader.  Adobe representatives have commented that these
%% fonts can be used outside of Adobe Reader.
%\setromanfont[Numbers=OldStyle]{Minion Pro}
%\setsansfont[Numbers=OldStyle,Scale=MatchLowercase]{Myriad Pro}
%\setmonofont[Scale=MatchLowercase]{Andale Mono}

%% Other alternatives:
%\setromanfont[Mapping=tex-text]{Adobe Caslon}
%\setsansfont[Scale=MatchLowercase]{Gill Sans}
%\setsansfont[Scale=MatchLowercase,Mapping=tex-text]{Futura}
%\setmonofont[Scale=MatchLowercase]{Andale Mono}
%\newfontfamily{\SYM}[Scale=0.9]{Zapf Dingbats}
%% END FONTS
%%%%%%%%%%%%%%%%%%%%%%%%%%%%%%%%%%%%%%%%%%%%%%%%%%%%%%%%%%%%%%%%%%%%%%
%%%%%%%%%%%%%%%%%%%%%%%%%%%%%%%%%%%%%%%%%%%%%%%%%%%%%%%%%%%%%%%%%%%%%%



%%%%%%%%%%%%%%%%%%%%%%%%%%%%%%%%%%%%%%%%%%%%%%%%%%%%%%%%%%%%%%%%%%%%%%
%%%%%%%%%%%%%%%%%%%%%%%%%%%%%%%%%%%%%%%%%%%%%%%%%%%%%%%%%%%%%%%%%%%%%%
%%
%% Recommended packages
%%
\usepackage{checkend}	% better error messages on left-open environments
\usepackage{graphicx}	% for incorporating external images

%% booktabs: provides some special commands for typesetting tables as used
%% in excellent journals.  Ignore the examples in the Lamport book!
\usepackage{booktabs}

%% listings: useful support for including source code listings, with
%% optional special keyword formatting.  The \lstset{} causes
%% the text to be typeset in a smaller sans serif font, with
%% proportional spacing.
\usepackage{listings}
\lstset{basicstyle=\sffamily\scriptsize,showstringspaces=false,fontadjust}

%% The acronym package provides support for defining acronyms, providing
%% their expansion when first used, and building glossaries.  See the
%% example in glossary.tex and the example usage throughout the example
%% document.
%% NOTE: to use \MakeTextLowercase in the \acsfont command below,
%%   we *must* use the `nohyperlinks' option -- it causes errors with
%%   hyperref otherwise.  See Section 5.2 in the ``LaTeX 2e for Class
%%   and Package Writers Guide'' (clsguide.pdf) for details.
\usepackage[printonlyused,nohyperlinks]{acronym}
%% The ubcdiss.cls loads the `textcase' package which provides commands
%% for upper-casing and lower-casing text.  The following causes
%% the acronym package to typeset acronyms in small-caps
%% as recommended by Bringhurst.
\renewcommand{\acsfont}[1]{{\scshape \MakeTextLowercase{#1}}}

%% color: add support for expressing colour models.  Grey can be used
%% to great effect to emphasize other parts of a graphic or text.
%% For an excellent set of examples, see Tufte's "Visual Display of
%% Quantitative Information" or "Envisioning Information".
\usepackage{color}
\definecolor{greytext}{gray}{0.5}

%% comment: provides a new {comment} environment: all text inside the
%% environment is ignored.
%%   \begin{comment} ignored text ... \end{comment}
\usepackage{comment}

%% The natbib package provides more sophisticated citing commands
%% such as \citeauthor{} to provide the author names of a work,
%% \citet{} to produce an author-and-reference citation,
%% \citep{} to produce a parenthetical citation.
%% We use \citeeg{} to provide examples
\usepackage[numbers,sort&compress]{natbib}
\newcommand{\citeeg}[1]{\citep[e.g.,][]{#1}}

%% The titlesec package provides commands to vary how chapter and
%% section titles are typeset.  The following uses more compact
%% spacings above and below the title.  The titleformat that follow
%% ensure chapter/section titles are set in singlespace.
\usepackage[compact]{titlesec}
\titleformat*{\section}{\singlespacing\raggedright\bfseries\Large}
\titleformat*{\subsection}{\singlespacing\raggedright\bfseries\large}
\titleformat*{\subsubsection}{\singlespacing\raggedright\bfseries}
\titleformat*{\paragraph}{\singlespacing\raggedright\itshape}

%% The caption package provides support for varying how table and
%% figure captions are typeset.
\usepackage[format=hang,indention=-1cm,labelfont={bf},margin=1em]{caption}

%% url: for typesetting URLs and smart(er) hyphenation.
%% \url{http://...} 
\usepackage{url}
\urlstyle{sf}	% typeset urls in sans-serif


%%%%%%%%%%%%%%%%%%%%%%%%%%%%%%%%%%%%%%%%%%%%%%%%%%%%%%%%%%%%%%%%%%%%%%
%%%%%%%%%%%%%%%%%%%%%%%%%%%%%%%%%%%%%%%%%%%%%%%%%%%%%%%%%%%%%%%%%%%%%%
%%
%% Possibly useful packages: you may need to explicitly install
%% these from CTAN if they aren't part of your distribution;
%% teTeX seems to ship with a smaller base than MikTeX and MacTeX.
%%
%\usepackage{pdfpages}	% insert pages from other PDF files
%\usepackage{longtable}	% provide tables spanning multiple pages
%\usepackage{chngpage}	% support changing the page widths on demand
%\usepackage{tabularx}	% an enhanced tabular environment

%% enumitem: support pausing and resuming enumerate environments.
%\usepackage{enumitem}

%% rotating: provides two environments, sidewaystable and sidewaysfigure,
%% for typesetting tables and figures in landscape mode.  
%\usepackage{rotating}

%% subfig: provides for including subfigures within a figure,
%% and includes being able to separately reference the subfigures.
%\usepackage{subfig}

%% ragged2e: provides several new new commands \Centering, \RaggedLeft,
%% \RaggedRight and \justifying and new environments Center, FlushLeft,
%% FlushRight and justify, which set ragged text and are easily
%% configurable to allow hyphenation.
%\usepackage{ragged2e}

%% The ulem package provides a \sout{} for striking out text and
%% \xout for crossing out text.  The normalem and normalbf are
%% necessary as the package messes with the emphasis and bold fonts
%% otherwise.
%\usepackage[normalem,normalbf]{ulem}    % for \sout

%%%%%%%%%%%%%%%%%%%%%%%%%%%%%%%%%%%%%%%%%%%%%%%%%%%%%%%%%%%%%%%%%%%%%%
%% HYPERREF:
%% The hyperref package provides for embedding hyperlinks into your
%% document.  By default the table of contents, references, citations,
%% and footnotes are hyperlinked.
%%
%% Hyperref provides a very handy command for doing cross-references:
%% \autoref{}.  This is similar to \ref{} and \pageref{} except that
%% it automagically puts in the *type* of reference.  For example,
%% referencing a figure's label will put the text `Figure 3.4'.
%% And the text will be hyperlinked to the appropriate place in the
%% document.
%%
%% Generally hyperref should appear after most other packages

%% The following puts hyperlinks in very faint grey boxes.
%% The `pagebackref' causes the references in the bibliography to have
%% back-references to the citing page; `backref' puts the citing section
%% number.  See further below for other examples of using hyperref.
%% 2009/12/09: now use `linktocpage' (Jacek Kisynski): GPS now prefers
%%   that the ToC, LoF, LoT place the hyperlink on the page number,
%%   rather than the entry text.
\usepackage[bookmarks,bookmarksnumbered,%
    allbordercolors={0.8 0.8 0.8},%
    pagebackref,linktocpage%
    ]{hyperref}
%% The following change how the the back-references text is typeset in a
%% bibliography when `backref' or `pagebackref' are used
\renewcommand\backrefpagesname{\(\rightarrow\) pages}
\renewcommand\backref{\textcolor{greytext} \backrefpagesname\ }

%% The following uses most defaults, which causes hyperlinks to be
%% surrounded by colourful boxes; the colours are only visible in
%% PDFs and don't show up when printed:
%\usepackage[bookmarks,bookmarksnumbered]{hyperref}

%% The following disables the colourful boxes around hyperlinks.
%\usepackage[bookmarks,bookmarksnumbered,pdfborder={0 0 0}]{hyperref}

%% The following disables all hyperlinking, but still enabled use of
%% \autoref{}
%\usepackage[draft]{hyperref}

%% The following commands causes chapter and section references to
%% uppercase the part name.
\renewcommand{\chapterautorefname}{Chapter}
\renewcommand{\sectionautorefname}{Section}
\renewcommand{\subsectionautorefname}{Section}
\renewcommand{\subsubsectionautorefname}{Section}

%% If you have long page numbers (e.g., roman numbers in the 
%% preliminary pages for page 28 = xxviii), you might need to
%% uncomment the following and tweak the \@pnumwidth length
%% (default: 1.55em).  See the tocloft documentation at
%% http://www.ctan.org/tex-archive/macros/latex/contrib/tocloft/
% \makeatletter
% \renewcommand{\@pnumwidth}{3em}
% \makeatother

%%%%%%%%%%%%%%%%%%%%%%%%%%%%%%%%%%%%%%%%%%%%%%%%%%%%%%%%%%%%%%%%%%%%%%
%%%%%%%%%%%%%%%%%%%%%%%%%%%%%%%%%%%%%%%%%%%%%%%%%%%%%%%%%%%%%%%%%%%%%%
%%
%% Some special settings that controls how text is typeset
%%
% \raggedbottom		% pages don't have to line up nicely on the last line
% \sloppy		% be a bit more relaxed in inter-word spacing
% \clubpenalty=10000	% try harder to avoid orphans
% \widowpenalty=10000	% try harder to avoid widows
% \tolerance=1000

%% And include some of our own useful macros
% This file provides examples of some useful macros for typesetting
% dissertations.  None of the macros defined here are necessary beyond
% for the template documentation, so feel free to change, remove, and add
% your own definitions.
%
% We recommend that you define macros to separate the semantics
% of the things you write from how they are presented.  For example,
% you'll see definitions below for a macro \file{}: by using
% \file{} consistently in the text, we can change how filenames
% are typeset simply by changing the definition of \file{} in
% this file.
% 
%% The following is a directive for TeXShop to indicate the main file
%%!TEX root = diss.tex

\newcommand{\NA}{\textsc{n/a}}	% for "not applicable"
\newcommand{\eg}{e.g.,\ }	% proper form of examples (\eg a, b, c)
\newcommand{\ie}{i.e.,\ }	% proper form for that is (\ie a, b, c)
\newcommand{\etal}{\emph{et al}}

% Some useful macros for typesetting terms.
\newcommand{\file}[1]{\texttt{#1}}
\newcommand{\class}[1]{\texttt{#1}}
\newcommand{\latexpackage}[1]{\href{http://www.ctan.org/macros/latex/contrib/#1}{\texttt{#1}}}
\newcommand{\latexmiscpackage}[1]{\href{http://www.ctan.org/macros/latex/contrib/misc/#1.sty}{\texttt{#1}}}
\newcommand{\env}[1]{\texttt{#1}}
\newcommand{\BibTeX}{Bib\TeX}

% Define a command \doi{} to typeset a digital object identifier (DOI).
% Note: if the following definition raise an error, then you likely
% have an ancient version of url.sty.  Either find a more recent version
% (3.1 or later work fine) and simply copy it into this directory,  or
% comment out the following two lines and uncomment the third.
\DeclareUrlCommand\DOI{}
\newcommand{\doi}[1]{\href{http://dx.doi.org/#1}{\DOI{doi:#1}}}
%\newcommand{\doi}[1]{\href{http://dx.doi.org/#1}{doi:#1}}

% Useful macro to reference an online document with a hyperlink
% as well with the URL explicitly listed in a footnote
% #1: the URL
% #2: the anchoring text
\newcommand{\webref}[2]{\href{#1}{#2}\footnote{\url{#1}}}

% epigraph is a nice environment for typesetting quotations
\makeatletter
\newenvironment{epigraph}{%
	\begin{flushright}
	\begin{minipage}{\columnwidth-0.75in}
	\begin{flushright}
	\@ifundefined{singlespacing}{}{\singlespacing}%
    }{
	\end{flushright}
	\end{minipage}
	\end{flushright}}
\makeatother

% \FIXME{} is a useful macro for noting things needing to be changed.
% The following definition will also output a warning to the console
\newcommand{\FIXME}[1]{\typeout{**FIXME** #1}\textbf{[FIXME: #1]}}

% END


%%%%%%%%%%%%%%%%%%%%%%%%%%%%%%%%%%%%%%%%%%%%%%%%%%%%%%%%%%%%%%%%%%%%%%
%%%%%%%%%%%%%%%%%%%%%%%%%%%%%%%%%%%%%%%%%%%%%%%%%%%%%%%%%%%%%%%%%%%%%%
%%
%% Document meta-data: be sure to also change the \hypersetup information
%%

\title{On the Use of the \texttt{ubcdiss} Template}
%\subtitle{If you want a subtitle}

\author{Johnny Canuck}
\previousdegree{B. Basket Weaving, University of Illustrious Arts, 1991}
\previousdegree{M. Silly Walks, Another University, 1994}

% What is this dissertation for?
\degreetitle{Doctor of Philosophy}

\institution{The University Of British Columbia}
\campus{Vancouver}

\faculty{The Faculty of XXX}
\department{Basket Weaving}
\submissionmonth{April}
\submissionyear{2192}

%% hyperref package provides support for embedding meta-data in .PDF
%% files
\hypersetup{
  pdftitle={Change this title!  (DRAFT: \today)},
  pdfauthor={Johnny Canuck},
  pdfkeywords={Your keywords here}
}

%%%%%%%%%%%%%%%%%%%%%%%%%%%%%%%%%%%%%%%%%%%%%%%%%%%%%%%%%%%%%%%%%%%%%%
%%%%%%%%%%%%%%%%%%%%%%%%%%%%%%%%%%%%%%%%%%%%%%%%%%%%%%%%%%%%%%%%%%%%%%
%% 
%% The document content
%%

%% LaTeX's \includeonly commands causes any uses of \include{} to only
%% include files that are in the list.  This is helpful to produce
%% subsets of your thesis (e.g., for committee members who want to see
%% the dissertation chapter by chapter).  It also saves time by 
%% avoiding reprocessing the entire file.
%\includeonly{intro,conclusions}
%\includeonly{discussion}

\begin{document}

%%%%%%%%%%%%%%%%%%%%%%%%%%%%%%%%%%%%%%%%%%%%%%%%%%
%% From Thesis Components: Tradtional Thesis
%% <http://www.grad.ubc.ca/current-students/dissertation-thesis-preparation/order-components>

% Preliminary Pages (numbered in lower case Roman numerals)
%    1. Title page (mandatory)
\maketitle

%    2. Abstract (mandatory - maximum 350 words)
%% The following is a directive for TeXShop to indicate the main file
%%!TEX root = diss.tex

\chapter{Abstract}

Psychophysical studies of perceptual learning find that perceivers only improve their perception on stimuli similar to what they were trained on.
In contrast, speech perception studies of perceptual learning find generalization to novel contexts when words contain a modified sound to be learned.
This dissertation seeks to resolve the apparent conflict between the findings of these two research paradigms through incorporating attentional sets.
Attention can be oriented towards comprehension of the speaker's intended meaning or towards perception of a speaker's pronunciation.
Attention is proposed to affect perceptual learning as follows.
When attention is oriented towards comprehension, more abstract and less context-dependent representations are updated and the perceiver shows a generalized perceptual learning effect (seen in the speech perception literature).
When attention is oriented towards perception, more finely detailed and more context-dependent representations are updated and the perceiver shows a less generalized perceptual learning effect (seen in the psychophysics literature).
This proposal is supported by three experiments.
The first two implement a standard paradigm for perceptual learning in speech perception.
Promoting a more perception-oriented attentional set causes less generalized perceptual learning.
The final experiment uses a novel paradigm where modified sounds embedded in sentences as the exposure.
Perceptual learning is found only when the modified sound is embedded in words that are not predictable from the sentence.
When modified sounds are in predictable words, no perceptual learning is observed.
To account for this lack of perceptual learning, I hypothesize that sounds in predictable words are less reliable than sounds in words in isolation or unpredictable sentences, resulting in no perceptual learning.
Highly predictable sentences are more intelligible as a whole, but the individual words within them are less intelligible.
In the cases where perceptual learning is present, conditions where comprehension-oriented attentional sets are promoted show larger perceptual learning effects than conditions where perception-oriented attentional sets are promoted.
I argue that attention is a key component to the generalization of perceptual learning to new contexts.


% Consider placing version information if you circulate multiple drafts
%\vfill
%\begin{center}
%\begin{sf}
%\fbox{Revision: \today}
%\end{sf}
%\end{center}

\cleardoublepage

%    3. Preface
%% The following is a directive for TeXShop to indicate the main file
%%!TEX root = diss.tex

\chapter{Preface}

All of the work presented henceforth was conducted in the Speech in Context Laboratory at the University of British Columbia, Point Grey campus. 
All experiments and associated methods were approved by the University of British Columbia's Research Ethics Board [certificate \#H06-04047].

I was the lead investigator for all experiments.  
Jamie Russell and Jobie Hui aided in data collection for the experiments in Chapter 2.  
Jobie Hui and Michelle Chan were involved in stimulus preparation and data collection for the experiment in Chapter 3.
Molly Babel was involved throughout all experiments in concept formation and manuscript edits.
\cleardoublepage

%    4. Table of contents (mandatory - list all items in the preliminary pages
%    starting with the abstract, followed by chapter headings and
%    subheadings, bibliographies and appendices)
\tableofcontents
\cleardoublepage	% required by tocloft package

%    5. List of tables (mandatory if thesis has tables)
\listoftables
\cleardoublepage	% required by tocloft package

%    6. List of figures (mandatory if thesis has figures)
\listoffigures
\cleardoublepage	% required by tocloft package

%    7. List of illustrations (mandatory if thesis has illustrations)
%    8. Lists of symbols, abbreviations or other (optional)

%    9. Glossary (optional)
%% The following is a directive for TeXShop to indicate the main file
%%!TEX root = diss.tex

\chapter{Glossary}

This glossary uses the handy \latexpackage{acroynym} package to automatically
maintain the glossary.  It uses the package's \texttt{printonlyused}
option to include only those acronyms explicitly referenced in the
\LaTeX\ source.

% use \acrodef to define an acronym, but no listing
\acrodef{UI}{user interface}
\acrodef{UBC}{University of British Columbia}

% The acronym environment will typeset only those acronyms that were
% *actually used* in the course of the document
\begin{acronym}[ANOVA]
\acro{ANOVA}[ANOVA]{Analysis of Variance\acroextra{, a set of
  statistical techniques to identify sources of variability between groups}}
\acro{API}{application programming interface}
\acro{CTAN}{\acroextra{The }Common \TeX\ Archive Network}
\acro{DOI}{Document Object Identifier\acroextra{ (see
    \url{http://doi.org})}}
\acro{GPS}[GPS]{Graduate and Postdoctoral Studies}
\acro{PDF}{Portable Document Format}
\acro{RCS}[RCS]{Revision control system\acroextra{, a software
    tool for tracking changes to a set of files}}
\acro{TLX}[TLX]{Task Load Index\acroextra{, an instrument for gauging
  the subjective mental workload experienced by a human in performing
  a task}}
\acro{UML}{Unified Modelling Language\acroextra{, a visual language
    for modelling the structure of software artefacts}}
\acro{URL}{Unique Resource Locator\acroextra{, used to describe a
    means for obtaining some resource on the world wide web}}
\acro{W3C}[W3C]{\acroextra{the }World Wide Web Consortium\acroextra{,
    the standards body for web technologies}}
\acro{XML}{Extensible Markup Language}
\end{acronym}

% You can also use \newacro{}{} to only define acronyms
% but without explictly creating a glossary
% 
% \newacro{ANOVA}[ANOVA]{Analysis of Variance\acroextra{, a set of
%   statistical techniques to identify sources of variability between groups.}}
% \newacro{API}[API]{application programming interface}
% \newacro{GOMS}[GOMS]{Goals, Operators, Methods, and Selection\acroextra{,
%   a framework for usability analysis.}}
% \newacro{TLX}[TLX]{Task Load Index\acroextra{, an instrument for gauging
%   the subjective mental workload experienced by a human in performing
%   a task.}}
% \newacro{UI}[UI]{user interface}
% \newacro{UML}[UML]{Unified Modelling Language}
% \newacro{W3C}[W3C]{World Wide Web Consortium}
% \newacro{XML}[XML]{Extensible Markup Language}
	% always input, since other macros may rely on it

\textspacing		% begin one-half or double spacing

%   10. Acknowledgements (optional)
%% The following is a directive for TeXShop to indicate the main file
%%!TEX root = diss.tex

\chapter{Acknowledgments}

Thank those people who helped you. 
Molly
Jamie
Jobie
Michelle

Don't forget your parents or loved ones.

You may wish to acknowledge your funding sources.


%   11. Dedication (optional)

% Body of Thesis (not all sections may apply)
\mainmatter

\acresetall	% reset all acronyms used so far

%    1. Introduction
%% The following is a directive for TeXShop to indicate the main file
%%!TEX root = diss.tex

\chapter{Introduction}
\label{ch:Introduction}

\begin{epigraph}
    \emph{If I have seen farther it is by standing on the shoulders of
    Giants.} ---~Sir Isaac Newton (1855)
\end{epigraph}

This document provides a quick set of instructions for using the
\class{ubcdiss} class to write a dissertation in \LaTeX. 
Unfortunately this document cannot provide an introduction to using
\LaTeX.  The classic reference for learning \LaTeX\ is
\citeauthor{lamport-1994-ladps}'s
book~\cite{lamport-1994-ladps}.  There are also many freely-available
tutorials online;
\webref{http://www.andy-roberts.net/misc/latex/}{Andy Roberts' online
    \LaTeX\ tutorials}
seems to be excellent.
The source code for this docment, however, is intended to serve as
an example for creating a \LaTeX\ version of your dissertation.

We start by discussing organizational issues, such as splitting
your dissertation into multiple files, in
\autoref{sec:SuggestedThesisOrganization}.
We then cover the ease of managing cross-references in \LaTeX\ in
\autoref{sec:CrossReferences}.
We cover managing and using bibliographies with \BibTeX\ in
\autoref{sec:BibTeX}. 
We briefly describe typesetting attractive tables in
\autoref{sec:TypesettingTables}.
We briefly describe including external figures in
\autoref{sec:Graphics}, and using special characters and symbols
in \autoref{sec:SpecialSymbols}.
As it is often useful to track different versions of your dissertation,
we discuss revision control further in
\autoref{sec:DissertationRevisionControl}. 
We conclude with pointers to additional sources of information in
\autoref{sec:Conclusions}.

%%%%%%%%%%%%%%%%%%%%%%%%%%%%%%%%%%%%%%%%%%%%%%%%%%%%%%%%%%%%%%%%%%%%%%
\section{Suggested Thesis Organization}
\label{sec:SuggestedThesisOrganization}

The \acs{UBC} \acf{GPS} specifies a particular arrangement of the
components forming a thesis.\footnote{See
    \url{http://www.grad.ubc.ca/current-students/dissertation-thesis-preparation/order-components}}
This template reflects that arrangement.

In terms of writing your thesis, the recommended best practice for
organizing large documents in \LaTeX\ is to place each chapter in
a separate file.  These chapters are then included from the main
file through the use of \verb+\include{file}+.  A thesis might
be described as six files such as \file{intro.tex},
\file{relwork.tex}, \file{model.tex}, \file{eval.tex},
\file{discuss.tex}, and \file{concl.tex}.

We also encourage you to use macros for separating how something
will be typeset (\eg bold, or italics) from the meaning of that
something. 
For example, if you look at \file{intro.tex}, you will see repeated
uses of a macro \verb+\file{}+ to indicate file names.
The \verb+\file{}+ macro is defined in the file \file{macros.tex}.
The consistent use of \verb+\file{}+ throughout the text not only
indicates that the argument to the macro represents a file (providing
meaning or semantics), but also allows easily changing how
file names are typeset simply by changing the definition of the
\verb+\file{}+ macro.
\file{macros.tex} contains other useful macros for properly typesetting
things like the proper uses of the latinate \emph{exempli grati\={a}}
and \emph{id est} (\ie \verb+\eg+ and \verb+\ie+), 
web references with a footnoted \acs{URL} (\verb+\webref{url}{text}+),
as well as definitions specific to this documentation
(\verb+\latexpackage{}+).

%%%%%%%%%%%%%%%%%%%%%%%%%%%%%%%%%%%%%%%%%%%%%%%%%%%%%%%%%%%%%%%%%%%%%%
\section{Making Cross-References}
\label{sec:CrossReferences}

\LaTeX\ make managing cross-references easy, and the \latexpackage{hyperref}
package's\ \verb+\autoref{}+ command\footnote{%
    The \latexpackage{hyperref} package is included by default in this
    template.}
makes it easier still. 

A thing to be cross-referenced, such as a section, figure, or equation,
is \emph{labelled} using a unique, user-provided identifier, defined
using the \verb+\label{}+ command.  
The thing is referenced elsewhere using the \verb+\autoref{}+ command.
For example, this section was defined using:
\begin{lstlisting}
    \section{Making Cross-References}
    \label{sec:CrossReferences}
\end{lstlisting}
References to this section are made as follows:
\begin{lstlisting}
    We then cover the ease of managing cross-references in \LaTeX\
    in \autoref{sec:CrossReferences}.
\end{lstlisting}
\verb+\autoref{}+ takes care of determining the \emph{type} of the 
thing being referenced, so the example above is rendered as
\begin{quote}
    We then cover the ease of managing cross-references in \LaTeX\
    in \autoref{sec:CrossReferences}.
\end{quote}

The label is any simple sequence of characters, numbers, digits,
and some punctuation marks such as ``:'' and ``--''; there should
be no spaces.  Try to use a consistent key format: this simplifies
remembering how to make references.  This document uses a prefix
to indicate the type of the thing being referenced, such as \texttt{sec}
for sections, \texttt{fig} for figures, \texttt{tbl} for tables,
and \texttt{eqn} for equations.

For details on defining the text used to describe the type
of \emph{thing}, search \file{diss.tex} and the \latexpackage{hyperref}
documentation for \texttt{autorefname}.


%%%%%%%%%%%%%%%%%%%%%%%%%%%%%%%%%%%%%%%%%%%%%%%%%%%%%%%%%%%%%%%%%%%%%%
\section{Managing Bibliographies with \BibTeX}
\label{sec:BibTeX}

One of the primary benefits of using \LaTeX\ is its companion program,
\BibTeX, for managing bibliographies and citations.  Managing
bibliographies has three parts: (i) describing references,
(ii)~citing references, and (iii)~formatting cited references.

\subsection{Describing References}

\BibTeX\ defines a standard format for recording details about a
reference.  These references are recorded in a file with a
\file{.bib} extension.  \BibTeX\ supports a broad range of
references, such as books, articles, items in a conference proceedings,
chapters, technical reports, manuals, dissertations, and unpublished
manuscripts. 
A reference may include attributes such as the authors,
the title, the page numbers, the \ac{DOI}, or a \ac{URL}.  A reference
can also be augmented with personal attributes, such as a rating,
notes, or keywords.

Each reference must be described by a unique \emph{key}.\footnote{%
    Note that the citation keys are different from the reference
    identifiers as described in \autoref{sec:CrossReferences}.}
A key is a simple sequence of characters, numbers, digits, and some
punctuation marks such as ``:'' and ``--''; there should be no spaces. 
A consistent key format simiplifies remembering how to make references. 
For example:
\begin{quote}
   \fbox{\emph{last-name}}\texttt{-}\fbox{\emph{year}}\texttt{-}\fbox{\emph{contracted-title}}
\end{quote}
where \emph{last-name} represents the last name for the first author,
and \emph{contracted-title} is some meaningful contraction of the
title.  Then \citeauthor{kiczales-1997-aop}'s seminal article on
aspect-oriented programming~\cite{kiczales-1997-aop} (published in
\citeyear{kiczales-1997-aop}) might be given the key
\texttt{kiczales-1997-aop}.

An example of a \BibTeX\ \file{.bib} file is included as
\file{biblio.bib}.  A description of the format a \file{.bib}
file is beyond the scope of this document.  We instead encourage
you to use one of the several reference managers that support the
\BibTeX\ format such as
\webref{http://jabref.sourceforge.net}{JabRef} (multiple platforms) or
\webref{http://bibdesk.sourceforge.net}{BibDesk} (MacOS\,X only). 
These front ends are similar to reference manages such as
EndNote or RefWorks.


\subsection{Citing References}

Having described some references, we then need to cite them.  We
do this using a form of the \verb+\cite+ command.  For example:
\begin{lstlisting}
    \citet{kiczales-1997-aop} present examples of crosscutting 
    from programs written in several languages.
\end{lstlisting}
When processed, the \verb+\citet+ will cause the paper's authors
and a standardized reference to the paper to be inserted in the
document, and will also include a formatted citation for the paper
in the bibliography.  For example:
\begin{quote}
    \citet{kiczales-1997-aop} present examples of crosscutting 
    from programs written in several languages.
\end{quote}
There are several forms of the \verb+\cite+ command (provided
by the \latexpackage{natbib} package), as demonstrated in
\autoref{tbl:natbib:cite}.
Note that the form of the citation (numeric or author-year) depends
on the bibliography style (described in the next section).
The \verb+\citet+ variant is used when the author names form
an object in the sentence, whereas the \verb+\citep+ variant
is used for parenthetic references, more like an end-note.
\begin{table}
\caption{Available \texttt{cite} variants; the exact citation style
    depends on whether the bibliography style is numeric or author-year.}
\label{tbl:natbib:cite}
\centering
\begin{tabular}{lp{3.25in}}\toprule
Variant & Result \\
\midrule
% We cheat here to simulate the cite/citep/citet for APA-like styles
\verb+\cite+ & Parenthetical citation (\eg ``\cite{kiczales-1997-aop}''
    or ``(\citeauthor{kiczales-1997-aop} \citeyear{kiczales-1997-aop})'') \\
\verb+\citet+ & Textual citation: includes author (\eg
    ``\citet{kiczales-1997-aop}'' or
    or ``\citeauthor{kiczales-1997-aop} (\citeyear{kiczales-1997-aop})'') \\
\verb+\citet*+ & Textual citation with unabbreviated author list \\
\verb+\citealt+ & Like \verb+\citet+ but without parentheses \\
\verb+\citep+ & Parenthetical citation (\eg ``\cite{kiczales-1997-aop}''
    or ``(\citeauthor{kiczales-1997-aop} \citeyear{kiczales-1997-aop})'') \\
\verb+\citep*+ & Parenthetical citation with unabbreviated author list \\
\verb+\citealp+ & Like \verb+\citep+ but without parentheses \\
\verb+\citeauthor+ & Author only (\eg ``\citeauthor{kiczales-1997-aop}'') \\
\verb+\citeauthor*+ & Unabbreviated authors list 
    (\eg ``\citeauthor*{kiczales-1997-aop}'') \\
\verb+\citeyear+ & Year of citation (\eg ``\citeyear{kiczales-1997-aop}'') \\
\bottomrule
\end{tabular}
\end{table}

\subsection{Formatting Cited References}

\BibTeX\ separates the citing of a reference from how the cited
reference is formatted for a bibliography, specified with the
\verb+\bibliographystyle+ command. 
There are many varieties, such as \texttt{plainnat}, \texttt{abbrvnat},
\texttt{unsrtnat}, and \texttt{vancouver}.
This document was formatted with \texttt{abbrvnat}.
Look through your \TeX\ distribution for \file{.bst} files. 
Note that use of some \file{.bst} files do not emit all the information
necessary to properly use \verb+\citet{}+, \verb+\citep{}+,
\verb+\citeyear{}+, and \verb+\citeauthor{}+.

There are also packages available to place citations on a per-chapter
basis (\latexpackage{bibunits}), as footnotes (\latexpackage{footbib}),
and inline (\latexpackage{bibentry}).
Those who wish to exert maximum control over their bibliography
style should see the amazing \latexpackage{custom-bib} package.

%%%%%%%%%%%%%%%%%%%%%%%%%%%%%%%%%%%%%%%%%%%%%%%%%%%%%%%%%%%%%%%%%%%%%%
\section{Typesetting Tables}
\label{sec:TypesettingTables}

\citet{lamport-1994-ladps} made one grievous mistake
in \LaTeX: his suggested manner for typesetting tables produces
typographic abominations.  These suggestions have unfortunately
been replicated in most \LaTeX\ tutorials.  These
abominations are easily avoided simply by ignoring his examples
illustrating the use of horizontal and vertical rules (specifically
the use of \verb+\hline+ and \verb+|+) and using the
\latexpackage{booktabs} package instead.

The \latexpackage{booktabs} package helps produce tables in the form
used by most professionally-edited journals through the use of
three new types of dividing lines, or \emph{rules}.
% There are times that you don't want to use \autoref{}
Tables~\ref{tbl:natbib:cite} and~\ref{tbl:LaTeX:Symbols} are two
examples of tables typeset with the \latexpackage{booktabs} package.
The \latexpackage{booktabs} package provides three new commands
for producing rules:
\verb+\toprule+ for the rule to appear at the top of the table,
\verb+\midrule+ for the middle rule following the table header,
and \verb+\bottomrule+ for the bottom-most at the end of the table.
These rules differ by their weight (thickness) and the spacing before
and after.
A table is typeset in the following manner:
\begin{lstlisting}
    \begin{table}
    \caption{The caption for the table}
    \label{tbl:label}
    \centering
    \begin{tabular}{cc}
    \toprule
    Header & Elements \\
    \midrule
    Row 1 & Row 1 \\
    Row 2 & Row 2 \\
    % ... and on and on ...
    Row N & Row N \\
    \bottomrule
    \end{tabular}
    \end{table}
\end{lstlisting}
See the \latexpackage{booktabs} documentation for advice in dealing with
special cases, such as subheading rules, introducing extra space
for divisions, and interior rules.

%%%%%%%%%%%%%%%%%%%%%%%%%%%%%%%%%%%%%%%%%%%%%%%%%%%%%%%%%%%%%%%%%%%%%%
\section{Figures, Graphics, and Special Characters}
\label{sec:Graphics}

Most \LaTeX\ beginners find figures to be one of the more challenging
topics.  In \LaTeX, a figure is a \emph{floating element}, to be
placed where it best fits.
The user is not expected to concern him/herself with the placement
of the figure.  The figure should instead be labelled, and where
the figure is used, the text should use \verb+\autoref+ to reference
the figure's label.
\autoref{fig:latex-affirmation} is an example of a figure.
\begin{figure}
    \centering
    % For the sake of this example, we'll just use text
    %\includegraphics[width=3in]{file}
    \Huge{\textsf{\LaTeX\ Rocks!}}
    \caption{Proof of \LaTeX's amazing abilities}
    \label{fig:latex-affirmation}   % label should change
\end{figure}
A figure is generally included as follows:
\begin{lstlisting}
    \begin{figure}
    \centering
    \includegraphics[width=3in]{file}
    \caption{A useful caption}
    \label{fig:fig-label}   % label should change
    \end{figure}
\end{lstlisting}
There are three items of note:
\begin{enumerate}
\item External files are included using the \verb+\includegraphics+
    command.  This command is defined by the \latexpackage{graphicx} package
    and can often natively import graphics from a variety of formats.
    The set of formats supported depends on your \TeX\ command processor.
    Both \texttt{pdflatex} and \texttt{xelatex}, for example, can
    import \textsc{gif}, \textsc{jpg}, and \textsc{pdf}.  The plain
    version of \texttt{latex} only supports \textsc{eps} files.

\item The \verb+\caption+ provides a caption to the figure. 
    This caption is normally listed in the List of Figures; you
    can provide an alternative caption for the LoF by providing
    an optional argument to the \verb+\caption+ like so:
    \begin{lstlisting}
    \caption[nice shortened caption for LoF]{%
	longer detailed caption used for the figure}
    \end{lstlisting}
    \ac{GPS} generally prefers shortened single-line captions
    in the LoF: multiple-line captions are a bit unwieldy.

\item The \verb+\label+ command provides for associating a unique, user-defined,
    and descriptive identifier to the figure.  The figure can be
    can be referenced elsewhere in the text with this identifier
    as described in \autoref{sec:CrossReferences}.
\end{enumerate}
See Keith Reckdahl’s excellent guide for more details,
\webref{http://www.ctan.org/tex-archive/info/epslatex.pdf}{\emph{Using
imported graphics in LaTeX2e}}.

\section{Special Characters and Symbols}
\label{sec:SpecialSymbols}

\LaTeX\ appropriates many common symbols for its own purposes,
with some used for commands (\ie \verb+\{}&%+) and
mathematics (\ie \verb+$^_+), and others are automagically transformed
into typographically-preferred forms (\ie \verb+-`'+) or to
completely different forms (\ie \verb+<>+).
\autoref{tbl:LaTeX:Symbols} presents a list of common symbols and
their corresponding \LaTeX\ commands.  A much more comprehensive list 
of symbols and accented characters is available at:
\url{http://www.ctan.org/tex-archive/info/symbols/comprehensive/}
\begin{table}
\caption{Useful \LaTeX\ symbols}\label{tbl:LaTeX:Symbols}
\centering\begin{tabular}{ccp{0.5cm}cc}\toprule
\LaTeX & Result && \LaTeX & Result \\
\midrule
    \verb+\texttrademark+ & \texttrademark && \verb+\&+ & \& \\
    \verb+\textcopyright+ & \textcopyright && \verb+\{ \}+ & \{ \} \\
    \verb+\textregistered+ & \textregistered && \verb+\%+ & \% \\
    \verb+\textsection+ & \textsection && \verb+\verb!~!+ & \verb!~! \\
    \verb+\textdagger+ & \textdagger && \verb+\$+ & \$ \\
    \verb+\textdaggerdbl+ & \textdaggerdbl && \verb+\^{}+ & \^{} \\
    \verb+\textless+ & \textless && \verb+\_+ & \_ \\
    \verb+\textgreater+ & \textgreater && \\
\bottomrule
\end{tabular}
\end{table}

%%%%%%%%%%%%%%%%%%%%%%%%%%%%%%%%%%%%%%%%%%%%%%%%%%%%%%%%%%%%%%%%%%%%%%
\section{Changing Page Widths and Heights}

The \class{ubcdiss} class is based on the standard \LaTeX\ \class{book}
class that selects a line-width to carry approximately 66~characters
per line.  This character density is claimed to have a pleasing
appearance and also supports more rapid
reading~\cite{bringhurst-2002-teots}.  I would recommend that you
not change the line-widths!

\subsection{The \texttt{geometry} Package}

Some students are unfortunately saddled with misguided supervisors
or committee members whom believe that documents should have the
narrowest margins possible.  The \latexpackage{geometry} package is
helpful in such cases.  Using this package is as simple as:
\begin{lstlisting}
    \usepackage[margin=1.25in,top=1.25in,bottom=1.25in]{geometry}
\end{lstlisting}
You should check the package's documentation for more complex uses.

\subsection{Changing Page Layout Values By Hand}

There are some miserable students with requirements for page layouts
that vary throughout the document.  Unfortunately the
\latexpackage{geometry} can only be specified once, in the document's
preamble.  Such miserable students must set \LaTeX's layout parameters
by hand:
\begin{lstlisting}
    \setlength{\topmargin}{-.75in}
    \setlength{\headsep}{0.25in}
    \setlength{\headheight}{15pt}
    \setlength{\textheight}{9in}
    \setlength{\footskip}{0.25in}
    \setlength{\footheight}{15pt}

    % The *sidemargin values are relative to 1in; so the following
    % results in a 0.75 inch margin
    \setlength{\oddsidemargin}{-0.25in}
    \setlength{\evensidemargin}{-0.25in}
    \setlength{\textwidth}{7in}       % 1.1in margins (8.5-2*0.75)
\end{lstlisting}
These settings necessarily require assuming a particular page height
and width; in the above, the setting for \verb+\textwidth+ assumes
a \textsc{US} Letter with an 8.5'' width.
The \latexpackage{geometry} package simply uses the page height and
other specified values to derive the other layout values.
The
\href{http://tug.ctan.org/tex-archive/macros/latex/required/tools/layout.pdf}{\texttt{layout}}
package provides a
handy \verb+\layout+ command to show the current page layout
parameters. 


\subsection{Making Temporary Changes to Page Layout}

There are occasions where it becomes necessary to make temporary
changes to the page width, such as to accomodate a larger formula. 
The \latexmiscpackage{chngpage} package provides an \env{adjustwidth}
environment that does just this.  For example:
\begin{lstlisting}
    % Expand left and right margins by 0.75in
    \begin{adjustwidth}{-0.75in}{-0.75in}
    % Must adjust the perceived column width for LaTeX to get with it.
    \addtolength{\columnwidth}{1.5in}
    \[ an extra long math formula \]
    \end{adjustwidth}
\end{lstlisting}


%%%%%%%%%%%%%%%%%%%%%%%%%%%%%%%%%%%%%%%%%%%%%%%%%%%%%%%%%%%%%%%%%%%%%%
\section{Keeping Track of Versions with Revision Control}
\label{sec:DissertationRevisionControl}

Software engineers have used \acf{RCS} to track changes to their
software systems for decades.  These systems record the changes to
the source code along with context as to why the change was required.
These systems also support examining and reverting to particular
revisions from their system's past.

An \ac{RCS} can be used to keep track of changes to things other
than source code, such as your dissertation.  For example, it can
be useful to know exactly which revision of your dissertation was
sent to a particular committee member.  Or to recover an accidentally
deleted file, or a badly modified image.  With a revision control
system, you can tag or annotate the revision of your dissertation
that was sent to your committee, or when you incorporated changes
from your supervisor.

Unfortunately current revision control packages are not yet targetted
to non-developers.  But the Subversion project's
\webref{http://tortoisesvn.net/docs/release/TortoiseSVN_en/}{TortoiseSVN}
has greatly simplified using the Subversion revision control system
for Windows users.  You should consult your local geek.

A simpler alternative strategy is to create a GoogleMail account
and periodically mail yourself zipped copies of your dissertation.

%%%%%%%%%%%%%%%%%%%%%%%%%%%%%%%%%%%%%%%%%%%%%%%%%%%%%%%%%%%%%%%%%%%%%%
\section{Recommended Packages}

The real strength to \LaTeX\ is found in the myriad of free add-on
packages available for handling special formatting requirements.
In this section we list some helpful packages.

\subsection{Typesetting}

\begin{description}
\item[\latexpackage{enumitem}:]
    Supports pausing and resuming enumerate environments.

\item[\latexpackage{ulem}:]
    Provides two new commands for striking out and crossing out text
    (\verb+\sout{text}+ and \verb+\xout{text}+ respectively)
    The package should likely
    be used as follows:
    \begin{verbatim}
    \usepackage[normalem,normalbf]{ulem}
    \end{verbatim}
    to prevent the package from redefining the emphasis and bold fonts.

\item[\latexpackage{chngpage}:]
    Support changing the page widths on demand.

\item[\latexpackage{mhchem}:] 
    Support for typesetting chemical formulae and reaction equations.

\end{description}

Although not a package, the
\webref{http://www.ctan.org/tex-archive/support/latexdiff/}{\texttt{latexdiff}}
command is very useful for creating changebar'd versions of your
dissertation.


\subsection{Figures, Tables, and Document Extracts}

\begin{description}
\item[\latexpackage{pdfpages}:]
    Insert pages from other PDF files.  Allows referencing the extracted
    pages in the list of figures, adding labels to reference the page
    from elsewhere, and add borders to the pages.

\item[\latexpackage{subfig}:]
    Provides for including subfigures within a figure, and includes
    being able to separately reference the subfigures.  This is a
    replacement for the older \texttt{subfigure} environment.

\item[\latexpackage{rotating}:]
    Provides two environments, sidewaystable and sidewaysfigure,
    for typesetting tables and figures in landscape mode.  

\item[\latexpackage{longtable}:]
    Support for long tables that span multiple pages.

\item[\latexpackage{tabularx}:]
    Provides an enhanced tabular environment with auto-sizing columns.

\item[\latexpackage{ragged2e}:]
    Provides several new commands for setting ragged text (\eg forms
    of centered or flushed text) that can be used in tabular
    environments and that support hyphenation.

\end{description}


\subsection{Bibliography Related Packages}

\begin{description}
\item[\latexpackage{bibunits}:]
    Support having per-chapter bibliographies.

\item[\latexpackage{footbib}:]
    Cause cited works to be rendered using footnotes.

\item[\latexpackage{bibentry}:] 
    Support placing the details of a cited work in-line.

\item[\latexpackage{custom-bib}:]
    Generate a custom style for your bibliography.

\end{description}


%%%%%%%%%%%%%%%%%%%%%%%%%%%%%%%%%%%%%%%%%%%%%%%%%%%%%%%%%%%%%%%%%%%%%%
\section{Moving On}
\label{sec:Conclusions}

At this point, you should be ready to go.  Other handy web resources:
\begin{itemize}
\item \webref{http://www.ctan.org}{\ac{CTAN}} is \emph{the} comprehensive
    archive site for all things related to \TeX\ and \LaTeX. 
    Should you have some particular requirement, somebody else is
    almost certainly to have had the same requirement before you,
    and the solution will be found on \ac{CTAN}.  The links to
    various packages in this document are all to \ac{CTAN}.

\item An online
    \webref{http://www.ctan.org/get/info/latex2e-help-texinfo/latex2e.html}{%
	reference to \LaTeX\ commands} provides a handy quick-reference
    to the standard \LaTeX\ commands.

\item The list of 
    \webref{http://www.tex.ac.uk/cgi-bin/texfaq2html?label=interruptlist}{%
	Frequently Asked Questions about \TeX\ and \LaTeX}
    can save you a huge amount of time in finding solutions to
    common problems.

\item The \webref{http://www.tug.org/tetex/tetex-texmfdist/doc/}{te\TeX\
    documentation guide} features a very handy list of the most useful
    packages for \LaTeX\ as found in \ac{CTAN}.

\item The
\webref{http://www.ctan.org/tex-archive/macros/latex/required/graphics/grfguide.pdf}{\texttt{color}}
    package, part of the graphics bundle, provides handy commands
    for changing text and background colours.  Simply changing
    text to various levels of grey can have a very 
    \textcolor{greytext}{dramatic effect}.


\item If you're really keen, you might want to join the
    \webref{http://www.tug.org}{\TeX\ Users Group}.

\end{itemize}

\endinput

Any text after an \endinput is ignored.
You could put scraps here or things in progress.


%    2. Main body
% Generally recommended to put each chapter into a separate file
%% The following is a directive for TeXShop to indicate the main file
%%!TEX root = diss.tex

\chapter{Introduction}

Listeners of a language are faced with a large degree of phonetic variability when interacting with their fellow language users.  
Speakers can have different sizes, different genders, and different backgrounds that make speech sound categories, at first blush, overlapping in distribution and hard to separate in acoustic dimensions.
In addition to properties of the speaker varying, a listener's attention or goals in an interaction can vary, such as paying attention to a non-native speaker or being distracted by planning upcoming utterances or by another task entirely.  
Despite variability on the part of both the listener and the speaker, listeners can interpret disparate and variable productions as belonging to a single word type or sound category, a phonemenon referred to as perceptual constancy in categorization studies \citep{Shankweiler1977, Kuhl1979} and as recognition equivalence in word recognition tasks \citep{Sumner2013}.
One of the processes for achieving this constancy is perceptual learning, whereby perceivers update context-dependent categories.
In the speech perception literature, perceptual learning or perceptual adaptation refers to the updating of distributions corresponding to a sound category, such as /s/ or /\textesh/, for a particular speaker after exposure to speech with a modification to the sound category's distribution \citep{Norris2003}.
Exposure to a speaker affects a listener's perceptual system for that speaker,  even after only a few tokens \citep{Vroomen2007, Kraljic2008} or within the span of a single utterance \citep{Ladefoged1957}.
Perceptual learning is not limited to single speakers, and exposure to multiple speakers sharing a non-native accent increases the intelligibility of new speakers with that accent \citep{Bradlow2008}.

%
%Perceptual learning has been found only when it is either lexically-guided \citep{Norris2003} or visually-guided \citep{Bertelson2003}.
%If the speaker characteristic to be learned is embedded in audio-only non-words, no perceptual learning occurs \citep{Norris2003}, suggesting that the lexical connections are crucial for updating distributions in perceptual learning.
%If a listener is to learn that a speaker produces their /s/ sounds with a more /\textesh/-like quality, some words may facilitate this learning more than others.  
%The position in the word that a critical sound occupies has an effect on a myriad of tasks, such as phoneme restoration, phoneme categorization \citep{Pitt2006}, and even lexical decision \citep{Pitt2012}.
%Increasing the bias towards a word 

However, perceptual learning, particularly in speech perception, relies on the linguistic system, which interacts with listener attention and signal properties.
This dissertation investigates the interaction between linguistic, attentional and acoustic factors on perceptual learning of the /s/ category of a speaker is modified to be more /\textesh/-like.
The key hypothesis being tested is that the more evidence a listener has toward a specific word, be it acoustic, lexical or semantic/sentential, the stronger the link between an ambiguous sound embedded in that word and a sound category will be, which will lead to greater observed perceptual learning effects.
Two linguistic factors will be manipulated that have been found to affect the categorization and discrimination of speech sounds embedded in words, namely lexical bias and semantic predictability.
According the hypothesis above, increasing linguistic expectations for the words that contain the target sounds will result in greater perceptual learning.
Attention will be manipulated through task demands focusing on word recognition and additional instructions to some listeners to pay attention to the specific sound category to be learned.
I predict that focusing attention on the sound category will result in less perceptual learning, as the same attention manipulation has led to lower word endorsement rates in previous studies \citep{Pitt2012}.
Finally, the acoustic tokens that listeners are exposed to are manipulated to expose listeners to /s/ categories that are closer or farther from a typical /s/ category.
I predict that exposure only to tokens farther from the typical category will result in less perceptual learning than exposure to tokens closer to the typical category.
Attention is predicted to interact with the other factors, because the attention manipulation draws the listener's attention to the /s/ sounds, which may occur naturally in some conditions, but not others.
Lower lexical bias positions, such as the beginnings of words, are salient positions in linguistic theory. 
Productions farther from the typical production are more likely to be noticed, such as in phoneme restoration tasks where restorations are more likely when the replacing sound is similar to the restored sound \citep{Samuel1981}.
In both of these cases, increasing external attention on the /s/ category will likely not produce an effect beyond what attention the signal itself draws.

The structure of the thesis is as follows.
This chapter will review the recent literature on perceptual learning in speech perception (Section~\ref{sec:perceptuallearning}), as well as literature on the linguistic (Section~\ref{sec:linguistic}), attentional (Section~\ref{sec:attention}), and signal (Section~\ref{sec:signal}) factors manipulated in the three experiments of this dissertation.
Chapter~\ref{chap:lexdec} will detail two experiments using a lexically-guided perceptual learning paradigm, each with different conditions for levels of lexical bias and attention.  
The two experiments differ in the acoustic properties of the exposure tokens, with the first experiment using an /s/ category that is halfway between /s/ and /\textesh/, and the second experiment using an /s/ category that is more /\textesh/-like than /s/-like.  
Chapter~\ref{chap:sent} details an experiment using a novel perceptual learning paradigm that manipulates an additional linguistic factor, namely semantic predictability, to increase the linguistic expectations during exposure.
The perceptual learning literature has generally used consistent processing conditions to elicit perceptual learning effects, and a goal of this dissertation is to examine the robustness and degree of perceptual learning across different processing conditions.

\section{Perceptual learning}
\label{sec:perceptuallearning}

Perceptual learning is a well established phenomenon in the psychology and psychophysics literature. 
Training can improve a participants ability to discriminate in many disparate modalities, such as visual acuity, somatosensory spatial resolution, weight estimation, and discrimination of hue and acoustic pitch \citep[for review]{Gibson1953}. 
In this literature, perceptual learning is the improvement of a perceiver to judge the physical characteristics of objects in the world through training that assumes attention on the task, but doesn't require reinforcement, correction or reward.
This definition of perceptual learning corresponds more to what is termed ``selective adaptation'' in the speech perception literature rather than what is termed ``perceptual learning'', ``perceptual adaptation'' or ``perceptual recalibration.''  
In speech perception, selective adaptation is the phenomenon where listeners that are exposed repeatedly to a narrow distribution for a sound category, narrow their own perceptual category, resulting in a change in variance of the category, not of the mean of the category (along some acoustic-phonetic dimension) \citep{Eimas1973,Samuel1986,Vroomen2007}.
Perceptual learning or recalibration in the speech perception literature is a more broad updating of perceptual categories, either in mean or variance \citep{Norris2003, Vroomen2007}.

\citet{Norris2003} began the recent set of investigations into lexically-guided perceptual learning in speech.
\citet{Norris2003} exposed one group of Dutch listeners to a fricative halfway between /s/ and /f/ at the ends of words like \emph{olif} "olive" and \emph{radijs} "radish", while exposing another group to the ambiguous fricative at the ends of nonwords, like \emph{blif} and \emph{blis}.
Following exposure, both groups of listeners were tested on their categorization on a fricative continuum from 100\% /s/ to 100\% /f/. 
Listeners exposed to the ambiguous fricative at the end of words shifted their categorization behaviour, while those exposed to them at the end of nonwords did not.  The exposure using words was further differentiated by the bias introduced by the words.  
Half the tokens ending in the ambiguous fricative formed a word if the fricative was interpreted as /s/ but not if it was interpreted as /f/, and the others were the reverse.  
Listeners exposed only to the /s/-biased tokens categorized more of the /f/-/s/ continuum as /s/, and listeners exposed to /f/-biased tokens categorized more of the continuum as /f/.  
The ambiguous fricative was associated with either /s/ or /f/ dependent on the bias of the word, which led to an expanded category for that fricative at the expense of the other category.
These results crucially show that perceptual categories in speech are malleable to new exposure, and that the linguistic system of the listener facilitates generalization to that category in new forms and contexts.

In addition to lexically-guided perceptual learning, unambiguous visual cues to sound identity can cause perceptual learning as well; this is referred to as perceptual recalibration in that literature.
In \citet{Bertelson2003}, an auditory continuum from /aba/ to /ada/ was synthesized and paired with a video of a speaker producing /aba/ and a video of /ada/.  
Participants first completed a pretest that identified the maximally ambiguous step of the /aba/-/ada/ auditory continuum. 
 In eight blocks, participants were randomly exposed to the ambiguous auditory token paired with video for /aba/ or with the video for /ada/.  Following each block, they completed a short categorization test.  
Participants showed perceptual learning effects, such that they were more likely to respond with /aba/ if they had been exposed to video of /aba/ paired with the ambiguous token in the preceding block, and likewise for /ada/.

\citet{vanLinden2007} compared the perceptual recalibration effects from the visual lipread paradigm \citep{Bertelson2003} to the standard lexically-guided perceptual learning paradigm \citep{Norris2003}.  
Lipread recalibration and perceptual learning effects had comparable size, lasted equally as long, were enhanced when presented with a contrasting sound, and were both unaffected by periods of silence between exposure and categorization.  
The effects did not last through prolonged testing in this study, unlike in other studies \citep{Kraljic2005,Eisner2006}.
In those studies, lexically-guided perceptual learning effects persisted through intermediate tasks, as long as the task did not involve any contradictory evidence of the trait learned \citep{Kraljic2005}, and also persisted across 12 hours \citep{Eisner2006}.

Perceptual learning in speech perception can be captured well in terms of Bayesian belief updating \citep{Kleinschmidt2011} or as part of a predictive coding model of the brain \citep{Clark2013}.  
In Bayesian belief updating, the model categorizes the incoming stimuli based on multimodel cues, and then updates the distribution to reflect that categorization.  
This updated conditional distribution is then used for future categorizations in an iterative process.  
\citet{Kleinschmidt2011} model the results of the behavioural study in \citet{Vroomen2007} in a Bayesian framework, with models fit to each participant capturing the perceptual recalibration and selective adaptation shown by the participants over the course of the experiment.  
A similar, but more broad, framework is that of the predictive brain \citep{Clark2013}. 
This framework uses a hierarchical generative model that aims to minimize prediction error between bottom-up sensory inputs and top-down expectations.  
Mismatches between the top-down expectations and the bottom-up signals generate error signals that are used to modify future expectations.  
Perceptual learning then is the result of modifying expectations to match learned input.

Perceptual learning in the psychophysics literature has shown a large degree of exposure-specificity, where observers only show learning effects on the same or very similar stimuli as those they were trained on. As such, perceptual learning has been argued to reside or affect the early sensory pathways, where stimuli are represented with the greatest detail \citep{Gilbert2001}.  Perceptual learning in speech perception has also shown a large degree of exposure-specificity, where participants do not generalize cues across speech sounds \citep{Reinisch2014} or across speakers unless the sounds are similar across exposure and testing \citep{Eisner2005, Kraljic2005, Kraljic2007, Reinisch2013a}.  
On the other hand, lexically-guided perceptual learning in speech has shown a greater degree of generalization than would be expected from a purely psychophysical standpoint.  
The testing stimuli are generally quite different from the exposure stimuli, with participants exposed to multisyllabic words ending in an ambiguous sound and tested on monosyllabic words \citep{Reinisch2013} and nonwords \citep{Norris2003, Kraljic2005}, though exposure-specificity is found when exposure and testing use different positional allophones \citep{Mitterer2013}.

Why is lexically-guided perceptual learning more context-general?
The experiments performed in this dissertation will provide evidence that this context-generality can be manipulated through attentional sets of listeners.
A more comprehension-oriented attentional set, where a listener's goal is understand the meaning of speech, promotes generalization and leads to greater perceptual learning.  
A more perception-oriented attentional set, where a listener's goal is to perceive specific qualities of a signal, does not promote generalization.
These attentional sets will be explored in more detail in Section~\ref{sec:attention} following an examination of the linguistic factors that will be manipulated in the experiments in Chapters~\ref{chap:lexdec} and \ref{chap:sent}.
These linguistic factors serve to aid the listener in adopting comprehension-oriented attentional sets.

\section{Linguistic factors and perceptual learning}
\label{sec:linguistic}

The two linguistic factors manipulated in this dissertation are lexical bias and semantic predictability.  
Lexically-guided perceptual learning paradigms use lexical bias as the means to link an ambiguous sound to an unambiguous category.
In Chapter~\ref{chap:sent}, a novel, sententially-guided perceptual learning paradigm is used to manipulate the listener's linguistic expectations in a different manner than manipulating lexical bias.

\subsection{Lexical bias}
\label{sec:lexicalbias}

Lexical bias is the primary way through which perceptual learning is induced in the experimental speech perception literature.
Lexical bias, also known as the Ganong Effect, refers to the tendency for listeners to interpret a speaker's (noncanonical) production as a particular, meaningful word rather than a nonsense word.  
For instance, given a continuum from a nonword like \emph{dask} to word like \emph{task} that differs only in the initial sound, listeners in general are more likely to interpret any step along the continuum as the word endpoint rather than the nonword endpoint \citep{Ganong1980}. 
This bias is exploited in perceptual learning studies to allow for noncanonical, ambiguous productions of a sound to be linked to pre-existing sound categories.
Given that ambiguous productions must be associated with a word to induce lexically-guided perceptual learning \citep{Norris2003}, differing degrees of lexical bias could lead to differing degrees of perceptual learning, a prediction which will be tested in Chapter~\ref{chap:lexdec}.  
The stronger the lexical bias, the stronger the link will be between the ambiguous sound and the sound category, as mediated by the word.

Studies looking at lexical bias generally use a phoneme categorization task, where participants identify a sound at the beginning or end of a word from two possible options that vary in one dimension.  
For instance, given a continuum from /t/ to /d/, participants are asked to identify the sound at the beginning or end as either /t/ or /d/. 
Lexical bias effects are calculated based on two continua, where words are formed at opposite ends. 
For the /t/ to /d/ continua, one would form a word at the /t/ end, such as \emph{task} and one would form a word at the other end, such as \emph{dash}.  
Lexical bias effects are then calculated from the different categorization behaviour for these two continua.

%Word length
Lexical bias has been found to vary in strength according to several factors.  
First, the length of the word in syllables has a large effect on lexical bias, with longer words showing stronger lexical bias than shorter words \citep{Pitt2006}.  
Continua formed using trisyllabic words, such as \emph{establish} and \emph{malpractice}, were found to show consistently large lexical bias effects than monosyllabic words, such as \emph{kiss} and \emph{fish}.  
\citet{Pitt2006} also found that lexical bias from trisyllabic words was robust across experimental conditions, but lexical bias from monosyllabic words was more fragile and condition dependent.  
They argue that these affects arise from both the greater bottom-up information present in longer words and the greater lexical competition for shorter words.

%Position in the word
\citet{Pitt2012} used a lexical decision task with a continuum of fricatives from /s/ to /\textesh/ embedded in words differing in the position of a sibilant.  
They found that ambiguous fricatives earlier in the word, such as \emph{serenade} or \emph{chandelier}, lead to greater nonword responses than the same ambiguous fricatives embedded later in a word, such as \emph{establish} or \emph{embarass}.  
In the experiments and meta-analysis of phoneme identification results presented in \citet{Pitt1993}, they found that, for monosyllabic word frames, token-final targets produce more robust lexical bias effects than token-initial targets.
Between word length and position in the word, lexical bias appears to be strengthened over the course of the word.
As a listener hears more evidence for a particular word, their expectations for hearing the rest of that word increase.

%Samuel1981
Lexical bias has been found to affect phoneme restoration tasks as well \citep{Samuel1981}.  
In this paradigm, listeners heard words with noise added to or replacing sounds and were asked to identify which they had heard for each trial.  
Lower sensitivity to noise addition versus replacement and increased bias for responding that noise was added is indicative of phoneme restoration, where listeners are perceiving sounds not actually present in the signal.  
Several factors are identified in the study as increasing the likelihood of the phoneme restoration effect.
In the lexical domain, words are more likely than nonwords to have phoneme restorations, and this discrepency strengthens when listeners are primed with a form without noise before the trial.  
More frequent words were also more likely to exhibit phoneme restoration effects, and longer words also showed greater phoneme restoration effects.  
Position of the sound in the word also influenced the decision, with non-initial positions showing greater phoneme restoration effects. 
The other influences on phoneme restoration discussed in this paper, namely the signal properties and sentential context will be discussed in subsequent sections.

In the stimuli used in \citet{Norris2003} and most other lexically-guided perceptual learning experiments, lexical bias tends to be maximized by using multisyllabic words with the ambiguous sound at the end for exposure stimuli.  
Perceptual learning has also been found when the ambiguous stimuli is embedded earlier in the word, such as the onset of the final syllable \citep{Kraljic2005, Kraljic2008, Kraljic2008a} or the even the onset of the first syllable \citep{Clare2014}.  
In light of the findings in \citet{Pitt2012}, we would expect less lexical biases the earlier in the word those ambiguous sounds were heard, and therefore, I hypothesize, lower endorsement rates and smaller perceptual learning effect sizes.  
This prediction will be explicitly tested in Experiment 1 in Chapter~\ref{chap:lexdec}.

\subsection{Semantic predictability}
\label{sec:semanticpredictability}

The second type of linguistic expectation manipulation used in this dissertation is known as semantic predictability \citep{Kalikow1977}.
Sentences are semantically predictable when they contain words prior to the final word that points almost definitively to the identity of that final word.  
For instance, the sentence fragment \emph{The cow gave birth to the...} from \citet{Kalikow1977} is almost guaranteed to be completed with the word \emph{calf}.  
On the other hand, a fragment like \emph{She is glad Jane called about the...} is far from having a guaranteed completion, other than having the category of noun.

Despite the temporal and spectral reduction found in high predictability contexts \cite{Scarborough2010, Clopper2008}, high predictability sentences are generally more intelligible.
Sentences that form a semantically coherent whole have higher word identification rates across varying signal-to-noise ratios \citep{Kalikow1977}, which has been found across children and adults \citep{Fallon2002}, and across native monolingual and early bilingual listeners, but not late bilingual listeners \citep{Mayo1997}.
However, when words at the ends of predictive sentences are excised from their context, they tend to be less intelligible than words excised from non-predictive contexts \citep{Lieberman1963}.
Highly predictable sentences are more intelligible to native listeners in noise, even when signal enhancements are not made, though non-native listeners require both signal enhancements and high predictability together to see any benefit \cite{Bradlow2007}.

Two studies looking at similarities between lexical bias and semantic predictability found similar effects on phoneme categorization, though differing effects in reaction time \citep{Connine1987,Borsky1998}.
\citet{Connine1987} found evidence that semantic predictability operated similar to the ``postperceptual'' processes in \citet{Connine1987a}.
The perceptual process was lexical bias towards one endpoint of a continuum, and the postperceptual process was a monetary benefit for responding with one of the end points of a continuum \citep{Connine1987a}.
The difference between ``perceptual'' and ``postperceptual'' processes in those studies was primarily one of reaction time; both kinds of bias produced comparable shifts in categorization.
\citet{Borsky1998} attempted to replicate \citet{Connine1987} while removing potential confounds from the methodological design.  
In contrast to \citet{Connine1987}, they used only one voicing continuum (from \emph{goat} to \emph{coat}), embedded the acoustic target in the middle of the sentence rather than at the end, and only presented one instance of each sentence to each participant rather than all continuum steps for each sentence to each participant.  
The reaction time profile in their study more closely aligned to the profile of lexical bias in \citet{Connine1987a}, but again the shift in how the continuum was categorized aligned with the sentential context.

In phoneme restoration, higher semantic predictability has been found to bias listeners toward interpreting the stimuli as intact with noise rather than replaced with noise \citep{Samuel1981}.
This increased bias towards interpreting the stimuli as an intact word was also coupled with an increase in sensitivity between the two types of stimuli, which \citet{Samuel1981} suggests is the result of a lower cognitive load in predictable contexts, and therefore greater phonetic encoding is available \citep[see also][]{Mattys2011}.

The previous literature on semantic predictability has shown largely similar effects as lexical bias in how sounds are categorized and restored.  
From this, I hypothesize that increasing the expectations for a word through semantic predictability will increase the link between sound categories and words in a similar fashion as increasing lexical bias.
Listeners that are exposed to an /s/ category that is more /\textesh/-like only in words that are highly predictable from context should show larger perceptual learning effects than listeners exposed to the same category only in words that are unpredictable from context.
However, there may be an upper limit for listener expectations when both semantic predictability and lexical bias are high, as committing too much to a particular expectation could lead to garden path phenomena \citep{Levy2008}.
The effect of semantic predictability on perceptual learning will be explicitly tested in Chapter~\ref{chap:sent}.

\section{Attentional factors and perceptual learning}
\label{sec:attention}

Attention is a large topic of research in its own right, and this section only reviews literature that is directly relevant to perceptual learning.
Attention has been found to have a role on perceptual learning in the psychophysics literature.  
For instance, \citet{Ahissar1993} found that in general, attending to global features for detection (i.e., discriminating different orientations of arrays of lines) does not make participants better at using local features for detection (i.e., detection of a singleton that differs in angle in the same arrays of lines), and vice versa.  However, there was a small degree of local detection learned from global detection, with the singleton popping out from its field as highly salient.

Attentional sets are a widely used term in the attention literature.  
In the visual domain, attentional sets can refer to the strategies that the perceiver uses to perform a task.  
For instance, in a visual search task, colour, orientation, motion and size are the predominant strategies \citep{Wolfe2004}.  
Two broad categories of attentional sets are generally used.  
Focused sets direct attention to components of the sensory input, and diffuse sets direct attention to global properties of the sensory input.  
In the visual search literature, a focused attentional set is the feature search mode, which gives priority to a single feature, such as the colour of the target, and a diffuse attentional set is singleton detection mode, which gives priority to any salient features \citep{Bacon1994}. 
In the auditory streaming literature, two attentional sets have been identified as ``selective listening'', where the perceiver attempts to hear the components of two streams, and ``comprehensive listening'', where the perceiver tries to hear all components as a single stream  \citep{vanNoorden1975}.
Finally, in a lexical decision tasks, the diffuse attentional set is where primary attention is on detecting words from nonwords, and the focused attentional set is where instructions direct participants attention to a potentially misleading sound \citep{Pitt2012}.
Attentional set selection is not necessarily optimal on the part of the perceiver, and it has been shown to be biased based on experience, with the amount of training performed influencing the length of time that perceivers will continue to use non-optimal sets after the task has changed \citep{Leber2006}.  

Listeners can selectively attend to many aspects of a signal, broadly falling into the categories of perception and comprehension.
As specified in \citet{Cutler1987}, perception refers to low-level, signal based attentional sets, while comprehension refers to lexical and other linguistically-based attentional sets.
For instance, listeners can attend to particular syllables or sounds in syllable or phoneme monitoring tasks \citep[and others]{Norris1988}, and even particular linguistically relevant temporal positions \citep{Pitt1990}.
However, even in these low-level, signal based tasks, lexical properties of the signal can exhibit some influence, if the stimuli are not monotonous enough to disengage comprehension \citep{Cutler1987}.  
Variation in speech in general seems to lead towards a more diffuse, comprehension-oriented attentional set, which is likely the default attentional set employed in everyday use of language, where the goal is firmly more comprehension than perception.

Attention has not been manipulated in previous work on perceptual learning in speech perception, but some work has been done on how individual differences in attention control can impact perceptual learning.
\citet{Scharenborg2014} presents a perceptual learning study of older Dutch listeners in the model of \citet{Norris2003}.  
In addition to the exposure and test phases, these older listeners completed tests for hearing loss, selective attention and attention-switching control.  
They found no evidence that perceptual learning was influenced by listeners' hearing loss or selective attention abilities, but they did find a significant relationship between a listener's attention-switching control and their perceptual learning.  
Listeners with worse attention-switching control showed greater perecptual learning effects, which the authors ascribed to an increased reliance on lexical information.  
Older listeners had previously been shown to have smaller perceptual learning effects as compared to younger listeners, but the differences were most prominent directly following exposure \citep{Scharenborg2013}.  
Younger listeners initially had a larger perceptual learning effect in the first block of testing, but the effect lessened over the subsequent blocks.  
Older listeners showed smaller initial perceptual learning effects, but no such decay.  
\citet{Scharenborg2013} also found that performance in the lexical decision task significantly affected the perceptual learning in the testing phase.

%Attention
Lexical bias is affected by attentional processing conditions.
\citet{Pitt2012} additionally investigated the role of attention in modulating lexical bias.  
When listeners were told that the speaker's /s/ and /\textesh/ were ambiguous and to listen carefully to ensure correct responses, they were less tolerant of noncanonical productions across all positions in the word.  
That is, participants attending to the speaker's sibilants were less likely to accept the modified production as a word than participants given no particular instructions about the sibilants.
Given that the task listeners performed was a lexical decision task, the default attentional set for the task, termed ``diffuse'' by \citet{Pitt2012}, would have attention distributed across both acoustic-phonetic and lexical domains.  Listeners given the instructions about the speaker's /s/ productions had a ``focused'' attentional set, with more weighting on the acoustic-phonetic domain than the ``diffuse'' attentional set.
Under higher cognitive load, such as performing a more difficult concurrent task, listeners show an increased lexical bias, as a result of weaker encoding of the auditory details \citep{Mattys2011}.  These results suggest that detailed encoding requires attentional resources.  In \citet{Mattys2011}, the primary task was a phoneme identification task, where a ``focused'' attentional set would likely be the default for participants, but a more ``diffuse'' attentional set, or one that weights lexical information more heavily, seems to be employed in the higher cognitive load conditions.

Attentional sets are thought to be constant across a task, but the fact that attention switching in older adults was a significant predictor of the size of perceptual learning effects \citep{Scharenborg2014} suggests that listeners are indeed switching sets through an experiment.  
The task is oriented toward comprehension, so the primary attentional set is likely to be a diffuse one relying more on lexical information than acoustic.
Participants with worse attention-switching control would have adopted this attentional set for more of the exposure than those with better attention-switching control, and it is precisely those with worse attention-switching control that showed the larger perceptual learning effects.
The ability to switch attention to a more perception-oriented set could have allowed those listeners prevent over-generalization, leading to a smaller perceptual learning effect for participants with better attention-switching control. 

In this dissertation, focusing a listener's attention on the signal through explicit instructions is hypothesized to lead to smaller perceptual learning effects.
In a focused perception-oriented attentional set, comprehension will be de-emphasized and the link between the sound category and the ambiguous token will be less mediated by lexical categories, leading to a more exposure-specific learning that will not generalize to novel tokens as readily.
The work on phoneme restoration suggests that higher predictability sentences are lower cognitive load than unpredictable sentences \citep{Samuel1981}, which may lead to greater encoding for the sounds to be learned, leading to larger perceptual learning effects.
However, if fewer attentional resources are required for comprehending the word, there could be more attention on perception in the predictable sentences, leading to a reduced perceptual effect.

\section{Signal factors and perceptual learning}
\label{sec:signal}

A primary finding across the perceptual learning literature is that learning effects are only found on testing items that are similar to the exposure items.  However, a less studied question is what properties of the exposure items cause different degrees of perceptual learning.  

Variability is a fundamental property of the speech signal, so sound categories must have some variance associated with them, and certain contexts can have increased degrees of variability.
\citet{Kraljic2008a} exposed participants to ambiguous sibilants between /s/ and /\textesh/ in two different contexts.  
In one, the ambiguous sibilants were intervocalic, and in the other, they occurred as part of a /str/ cluster.  
Participants exposed to the ambiguous sound intervocalically showed a perceptual learning effect, while those exposed to the sibilants in /str/ environments did not.  
The sibilant in /str/ often surfaces closer to [\textesh] in many varieties of English, due to coarticulatory effects from the other consonants in the cluster, but the coarticulatory effects for merging /s/ and /\textesh/ are much weaker in intervocalic position.  
They argue that the interpretation of the ambiguous sound is done in context of the surrounding sounds, and only when the pronunciation variant is unexplainable from context is the variant learned and attributed to the speaker\citet[see also][]{Kraljic2008}.  
In addition to a continuum of /asi/ to /a\textesh i/, they also tested a continuum from /astri/ to /a\textesh tri/, and found comparable perceptual learning effects across both continua for those exposed to the intervocalic ambiguous sibilants, but no perceptual learning effects on either continua for the other condition, showing a context insentivity absent in other studies.

%
\citet{Sumner2011} investigated whether differences in presentation order in a perceptual learning experiment led to different learning effects of the /b/-/p/ category boundary for a native French speaker of English.  
The presentation order that showed the greatest perecptual learning effects was the one where tokens started out close to what a listener would expect for the categories (English-like voice onset time for /b/ and /p/) and shifted over the course of the experiment to what the speaker's actual categories were (French-like voice onset time for /b/ and /p/), despite the fact that this presentation order is not anything like what a listener would normally encounter when interacting with a non-native speaker of English.  
The condition that mirrored the more normal course of non-native speaker pronunciation changes, starting as more French-like and ending as more English-like, did not produce significantly different behaviour than control participants.  
One explanation for these results are that a small difference between the listener's expectations and the input provides more robust learning, and the future input uses the updated expectations for future input, allowing for greater shifts over time through smaller shifts per trial, which aligns with some conceptualizations for exemplar model dynamics.
For instance, in the exemplar model proposed by \citet{Pierrehumbert2001}, only input similar to the learned distribution is used for updating that distribution, and input too ambiguous given learned distributions is discarded.

Additionally, in the phoneme restoration literature, sounds that are similar to the replacement sound are more likely to be restored \citep{Samuel1981}.  
When the replacement noise is white noise, fricatives and stops are more likely to be restored than vowels and liquids.
Acoustic signals that better match expectations are thus less likely to be noticed as atypical.

In this dissertation, the degree of typicality of the modified category is manipulated.  
In one case, the /s/ category for the speaker is maximally ambiguous between /s/ and /\textesh/, but in the other, the category is more like /\textesh/ than like /s/.
The maximally ambiguous category is less likely to be noticed as an atypical production than the more /\textesh/-like /s/ category.
I hypothesize that the maximally ambiguous category will provide stronger links between ambiguous acoustic tokens and the /s/ category, due to their higher degree of similarity to the expected, typical /s/ category.
As such, perceptual learning effects are predicted to be greater when the modified /s/ category is more like the expected /s/ category than the expected /\textesh/ category.

\section{Current contribution}

Perceptual learning in speech perception generalizes to new forms and contexts far more than would be expected from a purely psychophysical perspective \citep{Norris2003,Gilbert2001}.
This dissertation examines how different conditions for exposure affect generalization to subsequent categorization, leading to larger or smaller observed perceptual learning effects.
The primary hypothesis being tested is that the greater the link between the sound category and the modified acoustic tokens, mediated by lexical items, the greater the generalization to new tokens will be and the larger the perceptual learning effect will be.
Additionally, comprehension-oriented attentional sets are hypothesized to facilitate the link between the sound category and the modified tokens, while perception-oriented attentional sets are hypothesized to inhibit the link.
The specific predictions being tested are that increased expectation, either through lexical bias or semantic predictability, will increase this link and lead to greater perceptual learning effects.  
Directing attention to the specific sound category will shift the attentional set of participants to a more perception-oriented one, decreasing overall perceptual learning. 

In this dissertation,  I induce manipulations of lexical bias and attention in Experiments 1 and 2 (Chapter~\ref{chap:lexdec}), and manipulations of sentence predictability and attention in Experiment 3 (Chapter~\ref{chap:sent}). 
Lexical bias has been shown to affect phoneme categorization tasks \citep{Ganong1980}, and can be manipulated by position of the ambiguous sound in the word and attention \citep{Pitt2012}.  
Sentence predictability has likewise been found to affect phoneme categorization and phoneme restoration tasks similar to lexical bias \citep{Borsky1998, Samuel1981}, and can be manipulated by the preceding words in the sentence \citep{Kalikow1977}.  
In all experiments, attention will be either comprehension-oriented, with a focus on lexical recognition or identity, or perception-oriented, with a focus on the /s/ category that is atypical.

%% The following is a directive for TeXShop to indicate the main file
%%!TEX root = diss.tex

\chapter{Methodology}

Stuff I have done

%% The following is a directive for TeXShop to indicate the main file
%%!TEX root = diss.tex

\chapter{Results}

Stuff I have found

%% The following is a directive for TeXShop to indicate the main file
%%!TEX root = diss.tex

\chapter{Conclusions}


\section{Effect of increased bias}

Bias for a word was manipulated in two ways, through position of the ambiguous sound in the word and through sentential manipulations.  Increased lexical bias resulted in 


\section{Specificity versus generalization}

In Experiment 1, greater perceptual learning was shown by participants exposed to ambiguous sounds later in the words, not to those at the beginnings of words

And yet, the testing continua consisted of stimuli with the sibilant at the beginnings of words, which are more similar to the S-Initial words than the S-Final words

Attention removed this effect of position, with the perceptual learning effect identical across participants in the exposure condition when they were told to pay attention to the sibilant

\section{Distance to canonical}

In Experiment 2, there was no effect of attention or lexical bias on perceptual learning, with a stable effect present for all listeners

There are two potential explanations for the lack of effects:

One, the increased distance to the canonical production increased the salience of the production and drew the listener's attention to the ambiguous productions, resulting in a similar pattern for listeners in Experiment 1 in the Attention condition

Two, the productions farther from canonical produce a weaker effect on the updating of a listener's categories

\begin{itemize}
\item Predicted by a Pierrehumbert model

\item This is supported somewhat by the weaker correlation between word endorsement rate and crossover point found in Experiment 2
	
\item Also supported by the findings of \citet{Sumner2011} where the most perceptual learning was found when the categories begin like the listener expects and gradually shift toward the speaker's actual boundaries over the course of exposure
\end{itemize}



%    3. Notes
%    4. Footnotes

%    5. Bibliography
\begin{singlespace}
\raggedright
\bibliographystyle{abbrvnat}
\bibliography{biblio}
\end{singlespace}

\appendix
%    6. Appendices (including copies of all required UBC Research
%       Ethics Board's Certificates of Approval)
%\include{reb-coa}	% pdfpages is useful here
\chapter{Supporting Materials}

This would be any supporting material not central to the dissertation.
For example:
\begin{itemize}
\item Authorizations from Research Ethics Boards for the various
    experiments conducted during the course of research.
\item Copies of questionnaires and survey instruments.
\end{itemize}


Ambiguous stimuli steps for experiments

Experiment 1

\begin{tabular}

Type & Word & Step chosen & Proportion /s/ response \\
Initial & ceiling & 7 & 0.40 \\
Initial & celery & 7 & 0.30 \\
Initial & cement & 7 & 0.26 \\
Initial & ceremony & 7 & 0.44 \\
Initial & saddle & 8 & 0.25 \\
Initial & safari & 6 & 0.45 \\
Initial & sailboat & 7 & 0.35 \\
Initial & satellite & 7 & 0.45 \\
Initial & sector & 6 & 0.39 \\
Initial & seminar & 7 & 0.33 \\
Initial & settlement & 7 & 0.42 \\
Initial & sidewalk & 7 & 0.30 \\
Initial & silver & 7 & 0.21 \\
Initial & socket & 7 & 0.30 \\
Initial & sofa & 7 & 0.26 \\
Initial & submarine & 7 & 0.45 \\
Initial & sunroof & 6 & 0.39 \\
Initial & surfboard & 7 & 0.59 \\
Initial & syrup & 6 & 0.37 \\
Final & carousel & 7 & 0.45 \\
Final & castle & 7 & 0.50 \\
Final & concert & 7 & 0.53 \\
Final & croissant & 7 & 0.42 \\
Final & currency & 7 & 0.58 \\
Final & cursor & 8 & 0.53 \\
Final & curtsy & 8 & 0.40 \\
Final & dancer & 7 & 0.45 \\
Final & dinosaur & 7 & 0.50 \\
Final & faucet & 7 & 0.45 \\
Final & fossil & 8 & 0.30 \\
Final & galaxy & 9 & 0.47 \\
Final & medicine & 8 & 0.55 \\
Final & missile & 10 & 0.30 \\
Final & monsoon & 8 & 0.42 \\
Final & pencil & 7 & 0.45 \\
Final & pharmacy & 8 & 0.42 \\
Final & tassel & 8 & 0.35 \\
Final & taxi & 8 & 0.50 \\
Final & whistle & 7 & 0.58 \\
 &  & 7.26 & 0.41 \\

\end{tabular}

Experiment 2

\begin{tabular}

Type & Word & Step chosen & Proportion /s/ response \\
Initial & ceiling & 8 & 0.20 \\
Initial & celery & 7 & 0.30 \\
Initial & cement & 7 & 0.26 \\
Initial & ceremony & 8 & 0.39 \\
Initial & saddle & 8 & 0.25 \\
Initial & safari & 7 & 0.21 \\
Initial & sailboat & 7 & 0.35 \\
Initial & satellite & 8 & 0.30 \\
Initial & sector & 6 & 0.39 \\
Initial & seminar & 7 & 0.33 \\
Initial & settlement & 8 & 0.35 \\
Initial & sidewalk & 7 & 0.30 \\
Initial & silver & 7 & 0.21 \\
Initial & socket & 7 & 0.30 \\
Initial & sofa & 7 & 0.26 \\
Initial & submarine & 9 & 0.32 \\
Initial & sunroof & 6 & 0.39 \\
Initial & surfboard & 8 & 0.25 \\
Initial & syrup & 6 & 0.37 \\
Final & carousel & 8 & 0.25 \\
Final & castle & 9 & 0.25 \\
Final & concert & 10 & 0.30 \\
Final & croissant & 8 & 0.20 \\
Final & currency & 9 & 0.30 \\
Final & cursor & 11 & 0.30 \\
Final & curtsy & 9 & 0.26 \\
Final & dancer & 8 & 0.26 \\
Final & dinosaur & 9 & 0.39 \\
Final & faucet & 8 & 0.25 \\
Final & fossil & 8 & 0.30 \\
Final & galaxy & 10 & 0.26 \\
Final & medicine & 9 & 0.30 \\
Final & missile & 10 & 0.30 \\
Final & monsoon & 9 & 0.15 \\
Final & pencil & 8 & 0.37 \\
Final & pharmacy & 9 & 0.39 \\
Final & tassel & 8 & 0.35 \\
Final & taxi & 10 & 0.35 \\
Final & whistle & 9 & 0.35 \\
 &  & 8.1282051282 & 0.2978782426 \\

\end{tabular}

\backmatter
%    7. Index
% See the makeindex package: the following page provides a quick overview
% <http://www.image.ufl.edu/help/latex/latex_indexes.shtml>


\end{document}
