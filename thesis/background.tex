%% The following is a directive for TeXShop to indicate the main file
%%!TEX root = diss.tex

\chapter{Background}

\section{Perceptual learning}


Listeners of a language are faced with a large degree of variability when interacting with their fellow language users.  
Speakers can have different sizes, different genders, and different backgrounds that make speech sound categories, at first blush, overlapping in distribution and harder to separate in acoustic domains.
Despite this, listeners maintain a large degree of perceptual constancy, through speaker normalization and perceptual learning.  
Ambiguous vowel sounds can be mapped to different categories depending on the vowels of the preceding context \citep{Ladefoged1957}, where the previously-unheard ambiguous tokens is normalized to the vowel space present in the context.  
Perceptual learning refers to the updating of distributions corresponding to a sound category for a particular speaker.

\citet{Norris2003} began the investigation into perceptual learning in speech. Previous work has demonstrated perceptual learning in the visual domain (Color hue reference?) and (I think auditory as well).  
\citet{Norris2003} exposed Dutch listeners to a fricative halfway between /s/ and /f/ at the ends of words and nonwords and tested their categorization on a fricative continuum from 100\% /s/ to 100\% /f/. 
 Listeners exposed to the ambiguous fricative at the end of words shifted their categorization behaviour, while those exposed to them at the end of nonwords did not.  The exposure using words was further differentiated by the bias introduced by the words.  
Half the tokens ending in the ambiguous fricative formed a word if the fricative was interpreted as /s/ but not if it was interpreted as /f/, and the others were the reverse.  
Listeners exposed only to the /s/-biased tokens categorized more of the /f/-/s/ continuum as /s/, and listeners exposed to /f/-biased tokens categorized more of the continuum as /f/.  
The ambiguous fricative was associated with either /s/ or /f/ dependent on the bias of the word, which led to an expanded category for that fricative at the expense of the other category.

In addition to lexically-guided perceptual learning, unambiguous visual cues to sound identity can cause perceptual learning as well (referred to as perceptual recalibration in that literature). CITE.  Similarity to lexical guided perceptual learning (CITE).


\section{Rentention}

\citet{Kraljic2005} looked at the strength of the perceptual learning effect across various types of unlearning tasks. 
 Native English listeners were exposed to ambiguous fricatives in between /s/ and /\textesh/ in words that biased interpretation toward either /s/ or /\textesh/ in a lexical decision task.
 Participants that completed an /s/-/\textesh/ categorization task immediately following exposure and also those that had a 25 minute visua Unlearning task between exposure and categorization showed a large perceptual learning effect.  
Additionally, participants that had an Unlearning task involving auditory input from the same speaker as they were exposed to showed attenuated perceptual learning effects when the speech contained correct versions of the ambiguous fricatives, i.e. correct /s/ tokens when they were exposed to the ambiguous sibilant in all /s/ words in the exposure task.  
The perceptual learning effect in this condition was almost eliminated. 
 Participants that had an Unlearning task involving auditory input that did not have any /s/ or /\textesh/ tokens showed perceptual learning effects comparable to those with no Unlearning task.  
When the Unlearning task involved auditory input from a different speaker than exposure, perceptual learning effects were as robust as when there was no Unlearning task.

\citet{Eisner2006} looked at the stability of perceptual learning. 
Dutch participants first completed a categorization task for a continuum of /s/ to /f/ and then listened to a story that contained ambiguous fricative between /s/ and /f/, with one version having the ambiguous fricative in /s/ contexts and the other having it in /f/ contexts.
Immediately following, they completed another /s/ to /f/ categorization task.  
Finally, 12 hours later, they completed the categorization task once again.
Half the participants started at 9 am and completed the final categorization task at 9 pm the same day, and the other half started at 9 pm and completed the final categorization at 9 am the following day.
Perceptual learning effects were found in both categorization tasks following exposure, with no significant differences between the two times. 
 The perceptual learning effects found were robust across time and across intervening language exposure, as the those tested at 9 am and 9 pm likely had many interactions with other people in between.

Perceptual learning effects are robust across time.  
Given no other interactions with a given speaker, a listener's behaviour for that speaker's speech will remain constant over the course of a day, and probably longer. 
However, the perceptual system remains plastic.  
If a listener is exposed to a speaker trait in the first interaction, but the trait disappears in later interactions, the perceptual system will reattenuate to the newer evidence.

\section{Speaker dependency}

\citet{Kraljic2007} looked at different sound contrasts and the ability of participants to generalize across speakers.  
The first contrast was voicing between /t/ and /d/.  
When exposed to ambiguous stops between /t/ and /d/ corresponding to /d/ for one speaker, and ambiguous stops correpsonding to /t/ for another, perceptual learning effects corresponded to the most recent exposure, regardless of whether the test voice was the same as that recent exposure voice.  
The second contrast they used was sibilants of /s/ and /\textesh/. 
 For this contrast, they found the speaker-specificity effect found in the literature, where recency of the exposure did not affect the perceptual learning for a particular voice.  They argue that the differences between these two contrasts lies in the relative indexical information carried by the sibilants and the lack of much indexical information carried in the stops.

\citet{Eisner2005} looked at what manipulations could cause perceptual learning could generalize to additional talkers.  
Using a Dutch lexical decision task with ambiguous fricatives in between /s/ and /f/, they tested categorization on a contniuum between /s/ and /f/ from two talkers.  
When exposed to one talker's ambiguous fricative and tested on another speaker's continuum, no perceptual learning effect was found.  
There were two conditions where the fricatives in the exposure and categorization came from the same speaker, but the speakers of the carrier speech.  
In one, the ambiguous fricative from the categorization speaker's produtions was spliced into the exposure tokens, and in the other, the exposure talker's /s/ to /f/ continuum were spliced with vowels from the categorization talker.
In these two conditions, they did find a perceptual learning effect across speakers, despite reports by the participants of hearing different speakers, suggesting a sound-specificity effect in addition (or potentially causing) speaker-specificity effects. 

\citet{Reinisch2013} looked at perceptual learning of speaker characteristics across languages.  
When exposed to a native Dutch speaker speaking English with ambiguous fricatives between /s/ and /f/, participants show a perceptual learning effect when categorizing Dutch minimal pairs differing only in whether one sound is an /f/ or /s/ spoken by the same speaker.  
When exposed to the same native Dutch speaker speaking Dutch words and nonwords with the same ambiguous fricative, native Dutch listeners and L2 Dutch listeners (native German speakers) show comparable perceptual learning effects.

\section{Generalization}

\subsection{Generalizing to other categories}

\citet{Kraljic2006} looked at two sound contrasts (/d/-/t/ and /b/-/p/) that shared a feature, namely voicing.  
Participants were exposed to ambiguous in between /d/ and /t/ in English words and then tested on two continua with two different voices.  
The categorization continuum that was presented first, from /b/ to /p/, showed the stronger perceptual learning effects than the second continuum from /d/ to /t/, even though participants were exposed to ambiguous tokens between /d/ and /t/. 
 Both old voices and new voices for the categorization task produced perceptual learning effects.

\citet{Reinisch2011}

\citet{Reinisch2014}

\section{Timecourse}

\citet{Vroomen2004} \cite{Kleinschmidt2011}

\citet{Trude2012}

\section{Variation}

\citet{Sumner2011}


\section{Dependencies}

\subsection{Context specificity}

\citet{Kraljic2008a}

\citet{Kraljic2008}

\citet{Clare2014}