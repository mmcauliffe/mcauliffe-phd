%% The following is a directive for TeXShop to indicate the main file
%%!TEX root = diss.tex

\chapter{Background}

\section{Perceptual learning}

A listener's perceptual system shifts rapidly; for instance, vowel perception can shift solely on the basis of the vowels in a  preceding sentence \citep{Ladefoged1957}.  Boundaries between fricative categories can be changed by exposure to relatively small amount of ambiguous fricative tokens, but they must be embedded within and associated with a lexical item in a language, and hearing ambiguous productions between two sound categories in nonwords does not induce perceptual learning \citep{Norris2003}. However, besides the categorical distinction separating words from nonwords, little is known about the factors that influence linking an shifted or ambiguous production of a sound with a category through a lexical item. My dissertation investigates the factors which can contribute to associating ambiguous atypical productions with a specific sound category through lexical items. 


\citet{Norris2003}



\section{Rentention}

\citet{Kraljic2005}

\citet{Eisner2006}

\section{Speaker dependency}

\citet{Kraljic2007}

\citet{Reinisch2013}

\section{Timecourse}

\citet{Vroomen2004} \cite{Kleinschmidt2011}

\citet{Trude2012}

\section{Variation}

\citet{Sumner2011}

\section{Generalization}

Generalizing to other categories

\citet{Kraljic2006}

\citet{Reinisch2014}


\section{Dependencies}

Context specificity

\citet{Kraljic2008a}

\citet{Kraljic2008}

\citet{Reinisch2011}

\citet{Clare2014}
