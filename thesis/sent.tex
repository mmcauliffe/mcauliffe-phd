%% The following is a directive for TeXShop to indicate the main file
%%!TEX root = diss.tex

\chapter{Cross-modal word identification}

\section{Pre-test sentence completion experiment}

\section{Methodology}

\subsection{Participants}

Twenty participants completed a sentence predictability task before (following??) the s-sh categorization task described before. Participants were compensated with $10 CAD for both tasks, and were native North American English speakers with no reported speech, language or hearing disorders.

\subsection{Materials}

The sentences were written by me as either Predictive sentences or Unpredictive sentences.  Predictive sentences were ones where the final word was predictable based on the preceding sentence context, but in Unpredictive sentences, the final word was less or not predictable from the context.  Unlike previous studies using sentence or semantic predictability, Unpredictive sentences were written with the final word in mind with a variety of sentence structures, and the final words were plausible objects of lexical verbs and prepositions, unlike in previous work where verbs tended to be discussion words ("talk", "discuss", etc).  A full list of words and their contexts can be found in the SOMEWHERE.

\subsection{Procedure}

Participants first completed a sentence completion task.  In this task, participants were presented with sentence fragments that were lacking in the final word.  They were instructed to type in the word that came to mind when reading the fragment, and to enter any additional words that came to mind that would also complete the sentence.  There was no time limit for entry and participants were shown an example with the fragment "The boat sailed across the..." and the possible completions "bay, ocean, lake, river".

\subsection{Results}


\section{Experiment 3}

\section{Methodology}

\subsection{Materials}

\subsubsection{Exposure}

Participants were exposed to 100 sentences in the task.  All participants were exposed to the same 80 filler sentences, and differed in the 20 critical sentences they heard.  Sixty of the filler sentences had a final word that contained no sibilants  (/s, z, sh, zh, ch, jh/).  These final words were identical to the filler words used in experiments 1 and 2.  Twenty of the filler sentences ended in a word with a single /sh/ in the onset position of the final syllable, which matched the /sh/-words used in the Final-targets treatment in experiments 1 and 2.  Similarly, the critical sentences all ended in a word with a /s/ in the onset position of the final syllable, matching the /s/ words used in the Final-targets treatment in experiments.

Half of the filler items were classified as Unpredictive and half as Predictive.  Half of the participants were exposed to the target /s/ words in Predictive sentences, and half were exposed to target /s/ words in Unpredictive sentences.  

Pictures for the final words and pictures for distractors were selected by me from five candidates gathered by research assistants.  <TALK about impending picture naming task?>


\subsubsection{Categorization}

The categorization materials were identical to those used in Experiments 1 and 2.
	

\subsubsection{Synthesis}

All filler sentences were resynthesized using STRAIGHT.  For the critical /s/ tokens, a version of the sentence with a normal production of the final word, such as "At the carnival, the girl rode a unicorn around the carousel", was produced first, and then a version with /sh/ was produced, such as "At the carnival, the girl rode a unicorn around the caroushel", with the instructions that they be as similar in prosody and speech rate as possible.

To minimize any artifacts of the morphing procedure across the sentence, the sentence frame was kept as the first production with the natural /s/, and only the final word containing the sibilant was morphed.  The same step of the 11-step continuum that was identified as the 30\% point in the categorization pretesting for the individual words was used as the step for the ambiguous sentences as well.

\subsection{Participants}

Participants were assigned to one of four groups of 25 participants.  In the exposure phase, half of the participants were exposed to a modified /s/ sound only in Predictive sentences and half were exposed to it only in Unpredictive sentences.  Half of all participants were told that the speaker's production of "s" was somtimes ambiguous, and to listen carefully to ensure correct responses.  Participants were native North American English speakers with no reported speech or hearing disorders.

\subsection{Procedure}

For the exposure task, participants heard a sentence via headphones for each trial.  Immediately following the auditory presentation, they were presented with two pictures on the screen.  Their task was to select the picture on the screen that corresponded to the final word in the sentence they heard.

Following the exposure task, participants completed the same categorization task described in experiments 1 and 2.

