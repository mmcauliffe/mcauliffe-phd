%% The following is a directive for TeXShop to indicate the main file
%%!TEX root = diss.tex

\chapter{Cross-modal word identification}

\section{Methodology}

\subsection{Participants}

Participants were assigned to one of four groups of 25 participants.  In the exposure phase, half of the participants were exposed to a modified /s/ sound only in Predictive sentences and half were exposed to it only in Unpredictive sentences.  Half of all participants were told that the speaker's production of ``s'' was somtimes ambiguous, and to listen carefully to ensure correct responses.  Participants were native North American English speakers with no reported speech or hearing disorders.

\subsection{Materials}

\subsubsection{Exposure}

Participants were exposed to 100 sentences in the task.  All participants were exposed to the same 80 filler sentences, and differed in the 20 critical sentences they heard.  Sixty of the filler sentences had a final word that contained no sibilants  (/s, z, sh, zh, ch, jh/).  These final words were identical to the filler words used in experiments 1 and 2.  Twenty of the filler sentences ended in a word with a single /sh/ in the onset position of the final syllable, which matched the /sh/-words used in the Final-targets treatment in experiments 1 and 2.  Similarly, the critical sentences all ended in a word with a /s/ in the onset position of the final syllable, matching the /s/ words used in the Final-targets treatment in experiments.

Half of the filler items were classified as Unpredictive and half as Predictive.  Half of the participants were exposed to the target /s/ words in Predictive sentences, and half were exposed to target /s/ words in Unpredictive sentences.   Unlike previous studies using sentence or semantic predictability, Unpredictive sentences were written with the final word in mind with a variety of sentence structures, and the final words were plausible objects of lexical verbs and prepositions, unlike in previous work where verbs tended to be discussion words (``talk'', ``discuss'', etc).  A full list of words and their contexts can be found in the SOMEWHERE.

The same twenty participants that completed the lexical decision continua pre-test also completed a sentence predictability task before (following??) the s-sh categorization task described in Experiment 1. Participants were compensated with \$10 CAD for both tasks, and were native North American English speakers with no reported speech, language or hearing disorders. In this task, participants were presented with sentence fragments that were lacking in the final word.  They were instructed to type in the word that came to mind when reading the fragment, and to enter any additional words that came to mind that would also complete the sentence.  There was no time limit for entry and participants were shown an example with the fragment "The boat sailed across the..." and the possible completions "bay, ocean, lake, river".  RESULTS AND REPLACEMENTS

Pictures of 200 words, with 100 pictures for targets of the sentences and 100 for distractors, were selected in two steps.  First, a research assistant selected five images from a Google image search of the word, and then a single image representing that word was selected from amongst the five by me.  To ensure consistent behaviour in E-Prime, pictures were resized to fit within a 400x400 area with a resolution of 72x72 DPI and converted to bitmap format.  Additionally, any transparent backgrounds in the pictures were converted to plain white backgrounds.

Five volunteers from the Speech in Context lab participated in a pre-test to determine how suitable the pictures were at representing their associated word.  All participants were native speakers of North American English, with reported corrected-to-normal vision. Participants were presented with a single image in the middle of the screen.  Their task was to type the word that first came to mind, and any other words that described the picture equally well.  There was no time limit and presentation of the pictures was self-paced. Responses were sanitized by removing miscellaneous keystrokes recorded by E-Prime, spell checking, and standardizing variant spellings and plural forms.  Following sanitization, bags of words for each picture were constructed by combining all the responses by participants for each word.  From the bag of words, the response proportion was calculated as the count of the target word divided by all the words used by participants for that picture.  Additionally, a simpler measure was calculated as the proportion of participants that included the target word in their response. Five pictures were replaced, \emph{toothpick} and \emph{falafel} with clearer pictures and \emph{ukulele}, \emph{earmuff} and \emph{earplug} were replaced with ___, ___ and ___.  All five replacements were for distractor words.


\subsubsection{Categorization}

The categorization materials were identical to those used in Experiments 1 and 2 in the previous chapter.
	

\subsubsection{Synthesis}

All filler sentences were resynthesized using STRAIGHT.  For the critical /s/ tokens, a version of the sentence with a normal production of the final word, such as ``At the carnival, the girl rode a unicorn around the carousel'', was produced first, and then a version with /sh/ was produced, such as ``At the carnival, the girl rode a unicorn around the caroushel'', with the instructions that they be as similar in prosody and speech rate as possible.

To minimize any artifacts of the morphing procedure across the sentence, the sentence frame was kept as the first production with the natural /s/, and only the final word containing the sibilant was morphed.  The same step of the 11-step continuum that was identified as the 30\% point in the categorization pretesting for the individual words was used as the step for the ambiguous sentences as well.

\subsection{Procedure}

For the exposure task, participants heard a sentence via headphones for each trial.  Immediately following the auditory presentation, they were presented with two pictures on the screen.  Their task was to select the picture on the screen that corresponded to the final word in the sentence they heard.

Following the exposure task, participants completed the same categorization task described in experiments 1 and 2.

