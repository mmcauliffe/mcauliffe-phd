%% The following is a directive for TeXShop to indicate the main file
%%!TEX root = diss.tex

\chapter{Cross-modal word identification}

%This chapter investigates perceptual learning with a different bias than lexical, by using sentential context to bias participants towards a particular word.

\section{Motivation}

\subsection{Semantic predictability}

The type of contextual predictability used in this experiment is known as semantic predictability.  Sentences are highly predictable when they contain words prior to the final word that points almost definitively to the identity of the final word.  For instance, the sentence fragment \emph{The cow gave birth to the...} from \citet{Kalikow1977} is almost guaranteed to be completed with the word \emph{calf}.  On the other hand, a fragment like \emph{She is glad Jane called about the...} is far from having a guaranteed completion, other than having the category of noun. Semantic predictability has been found to have effects on both production and perception, which will be discussed in the two following sections

\subsubsection{Acoustic realization}

In general, semantic predictability has a reductive effect on the acoustics of words.  In \citet{Scarborough2010}, words produced in highly predictable frames tend to be shorter in duration and have less dispersed vowel realizations.  This semantic predictability effect did not interact with neighbourhood density, a lexical factor that proxies for the amount of lexical competition a word has.  Words with many neighbours, and therefore had lessened lexical predictability, had longer durations and more dispersed vowel realizations.  For both the lexical and the semantic predictability, high predictability led to less distinct word realizations, and low predictability led to more distinct word realizations, independently of each other.

In a study looking at semantic predictability across dialects, \citet{Clopper2008} found that not all dialects realize the effects of semantic predictability the same.  For the Southern dialect of American English, the results were much the same as in \citet{Scarborough2010}, showing temporal and spectral reduction in high predictabilty environments.  However, speakers in the Midland dialect showed no such effect, and speakers of the Northern dialect showed more extreme Northen Cities shifting in the the high predictability environment.

\subsubsection{Intelligibility}

Despite the temporal and spectral reduction found in high predictability contexts, high predictability sentences are generally more intelligible.  Sentences that form a semantically coherent whole have higher word identification rates across varying signal-to-noise ratios \citep{Kalikow1977}.  However, when words at the ends of predictive sentences are excised from their context, they tend to be less intelligible than words excised from non-predictive contexts \citep{Lieberman1963}.  <<I don't know if this is just due to differences in speaking rates/durations.  Do you know of any studies (maybe based on sentences excised from sponataneous speech) where words would be said with similar durations or have contextual frames that normalize the duration differences for the listener?>>

\subsection{Similarity to lexical bias}

A key pillar of the motivation underlying this experiment is the notion that semantic predictability is similar to, yet distinct from, lexical bias/lexical predictability.  \citet{Scarborough2010} found that lexical predictability and semantic predictability did not interact and had the same effect directions, but different effect sizes, with lexical predictability having a larger effect than semantic predictability.

One line of research has attempted to show whether the semantic predictability effects are part of perceptual and phonetic processing or if they result following phonetic processing.  \citet{Connine1987a} established different reaction time profiles for perceptual and postperceptual processes in categorizing a continuum.  With lexical bias, response times to the end points of a /d/ to /t/ continuum show no difference whether the continuum forms a word at one end point (such as \emph{dice} to \emph{tice}) or at the other (such as \emph{dype} to \emph{type}).  However, at the boundary between /d/ and /t/, reaction times are faster when a subject responds consisten to the bias (i.e. interprets an ambiguous word \emph{?ice} as \emph{dice}).  For postperceptual processes, in this case a monetary payoff for responding either /d/ or /t/ along a continuum that has nonwords at both ends, the pattern is reversed, where reaction times were faster for responses consistent with the monetary bias at the end points of the continuum, but no such difference was found for the category boundary.  Both biases produced similar categorization patterns, such that the participants biased toward /d/, either lexically or monetarily, categorized more of the continuum as /d/, so the principle difference between the two biases was in the reaction time profile.

Two studies that have looked at semantic predictability in this paradigm are \citet{Connine1987} and \citet{Borsky1998}, and they found conflicting results.  In \citet{Connine1987}, the reaction time profile aligned more with the monetary bias in \citet{Connine1987a}, with reaction time differences located at the end point and not the category boundary, but \citet{Borsky1998} found the reverse with a profile more similar to lexical bias.  <<Examine differences between studies?>>

\subsection{Attention?}

I don't know of any work that has specifically looked at attention (to acoustics in particular) in the context of semantic predictability.

\section{Methodology}

\subsection{Participants}

One hundred native speakers of English (mean age ??, range ??-??) participated in the experiment and were compensated with either \$10 CAD or course credit. They were recruited from the UBC student population.  Twenty additional native English speakers participated in a pretest to determine sentence predictability, and 10 other native English speakers participated in a picture naming pretest.

Participants were assigned to one of four groups of 25 participants.  In the exposure phase, half of the participants were exposed to a modified /s/ sound only in Predictive sentences and half were exposed to it only in Unpredictive sentences.  Half of all participants were told that the speaker's production of ``s'' was somtimes ambiguous, and to listen carefully to ensure correct responses.  Participants were native North American English speakers with no reported speech or hearing disorders.

\subsection{Materials}

One hundred and twenty sentences were used as exposure materials.  The set of sentences consisted of 40 critical sentences, 20 control sentences and 60 filler sentences.  The critical sentences ended in one of 20 of the critical words in Experiments 1 and 2 that had an /s/ in the onset of the final syllable.  The 20 control sentences ended in the 20 control items used in Experiments 1 and 2, and the 60 filler sentences ended in the 60 filler words in Experiments 1 and 2.  Half of all sentences were written to be predictive of the final word, and the other half were written to be unpredictive of the final word.  Unlike previous studies using sentence or semantic predictability \citep{Kalikow1977}, Unpredictive sentences were written with the final word in mind with a variety of sentence structures, and the final words were plausible objects of lexical verbs and prepositions.  A full list of words and their contexts can be found in the appendix. Aside from the sibilants in the critical and control words, the sentences contained no sibilants (/s z sh zh ch jh/).  The same minimal pairs for phonetic categorization as in Experiments 1 and 2 were used.

Sentences were recorded by the same male Vancouver English speaker used in Experiments 1 and 2.  Critical sentences were recorded in pairs, with one normal production and then a production of the same sentence with the /s/ in the final word replaced with an /sh/.  The speaker was instructed to produce both sentences with comparable speech rate, speech style and prosody.

As in Experiments 1 and 2, the critical items were morphed together into an 11-step continuum using STRAIGHT \citep{Kawahara2008}; however, only the final word in sentence was morphed.  For all steps, the preceding words in the sentence were kept as the natural production to minimize artifacts of the morphing algorithm.  As in Experiments 1 and 2, the control and filler items were processed and resynthesized.  The ambiguous point selection was based on the pretest performed for Experiment 1 and 2 exposure items.  The ambiguous steps of the continua chosen corresponded to the 50\% cross over point in Experiment 2.

Pictures of 200 words, with 100 pictures for the final word of the sentences and 100 for distractors, were selected in two steps.  First, a research assistant selected five images from a Google image search of the word, and then a single image representing that word was selected from amongst the five by me.  To ensure consistent behaviour in E-Prime, pictures were resized to fit within a 400x400 area with a resolution of 72x72 DPI and converted to bitmap format.  Additionally, any transparent backgrounds in the pictures were converted to plain white backgrounds.

\subsection{Pretest}

The same twenty participants that completed the lexical decision continua pre-test also completed a sentence predictability task before the phonetic categorization task described in Experiment 1. Participants were compensated with \$10 CAD for both tasks, and were native North American English speakers with no reported speech, language or hearing disorders. In this task, participants were presented with sentence fragments that were lacking in the final word.  They were instructed to type in the word that came to mind when reading the fragment, and to enter any additional words that came to mind that would also complete the sentence.  There was no time limit for entry and participants were shown an example with the fragment ``The boat sailed across the...'' and the possible completions ``bay, ocean, lake, river''.  Responses were collected in E-Prime (cite), and were sanitized by removing miscellaneous keystrokes recorded by E-Prime, spell checking, and standardizing variant spellings and plural forms.

The measure used for determining rewriting of sentences was the proportion of participants that included the target word in their responses.  For predictive sentences, the mean proportion was 0.49 (range 0-0.95) and for unpredictive sentences, the mean proportion was 0.03 (range 0-0.45).  Predictive sentences that had target response proportions of 20\% or less were rewritten.  The predictive sentences for \emph{auction}, \emph{brochure}, \emph{carousel}, \emph{cashier}, \emph{cockpit}, \emph{concert}, \emph{cowboy}, \emph{currency}, \emph{cursor}, \emph{cushion}, \emph{dryer}, \emph{graffiti}, and \emph{missile} were rewritten to remove any ambiguities.  

Five volunteers from the Speech in Context lab participated in another pretest to determine how suitable the pictures were at representing their associated word.  All participants were native speakers of North American English, with reported corrected-to-normal vision. Participants were presented with a single image in the middle of the screen.  Their task was to type the word that first came to mind, and any other words that described the picture equally well.  There was no time limit and presentation of the pictures was self-paced. Responses were sanitized as in the first pretest.  

Pictures were replaced if 20\% or less of the participants (1 of 5) responded with the target word and the responses were semantically unrelated to the target word. Five pictures were replaced, \emph{toothpick} and \emph{falafel} with clearer pictures and \emph{ukulele}, \emph{earmuff} and \emph{earplug} were replaced with \emph{rollerblader}, \emph{anchor} and \emph{bedroom}.  All five replacements were for distractor words.

\subsection{Procedure}

As in Experiments 1 and 2, participants completed two tasks, an exposure task and a categorization task.  For the exposure task, participants heard a sentence via headphones for each trial.  Immediately following the auditory presentation, they were presented with two pictures on the screen.  Their task was to select the picture on the screen that corresponded to the final word in the sentence they heard.  As in Experiments 1 and 2, the order was pseudorandom, with the same constraints.

Following the exposure task, participants completed the same categorization task described in Experiments 1 and 2.

\section{Results}

\section{Discussion}
