\documentclass[a4paper,11pt,twocolumn]{article}

\usepackage{icphs2015}
\usepackage{metalogo} % Only needed for the XeLaTeX logo

\title{Paper template for {ICPhS} 2015 Glasgow}
\author{Please write XXX instead of the name(s) of the author(s)}
\organization{Please write XXX instead of the affiliation(s)}
\email{please write XXX instead of the email address(es)}
\begin{document}

\maketitle

\begin{abstract}
This is the layout specification and template definition for the papers of ICPhS XVIII (the 18th International Congress of Phonetic Sciences, which will be held in the SECC, Glasgow, UK, August 10-14, 2015). This template is a revised version of the one used for Hong Kong ICPhS XVII in 2011, originally generated from the template for Speech Prosody 2006 in Dresden.

The abstract may consist of more than one paragraph but must be kept within a 150 word limit. This abstract will be printed in the abstract booklet to be given out at the conference.
\end{abstract}

\keywords{There is space for up to five self-selected keywords
  (maximally two lines).}


\section{Introduction}

The following rules apply to all submitted papers:

\begin{itemize}
\item they must be written in English
\item they must not contain the name(s) of the author(s) (for
anonymous review)
\item the maximum is four pages for Congress papers, plus up to one additional page of references. Papers for Plenary lectures can be longer
\item they must be submitted in PDF format (cf.\ Section~4)
\item the paper submission will occur via a web interface
\end{itemize}

This paper template can be found on the conference website. If there
are special questions or wishes regarding paper preparation and
submission for ICPhS 2015, correspondence should be addressed to
contact@icphs2015.info.  Information for full paper submission will be
available on the web at \url{http://www.icphs2015.info}.



\section{Page layout and style}

The page layout should conform to the following rules. By far the easiest way to meet these requirements is to use the supplied templates and check details against this example file. If for some reason you cannot use the template, please follow these rules as carefully as possible, or contact the editors at contact@icphs2015.info for further instructions.

\subsection{Basic layout features}

\begin{itemize}
\item The layout is appropriate for A4 format.
\item Two columns are used
  except for the title part and possibly for large figures that need a
  full page width.
\item Left margin is 20 mm. Right margin will depend on
  the size of the paper. Column width is 80 mm.
\item Spacing between
  columns is 10 mm.
\item Top margin 25 mm (except first page which has 30
  mm to the title top). Bottom margin will depend on the size of the
  paper.
\item Text height (without headers and footers) is a maximum of
  235 mm.  Headers and footers should be left empty
\item Check indentations
  and spacings by comparing with this example file.
\end{itemize}

\subsubsection{Headings}

Section headings are centered in boldface with capitalized letters. Sub-headings start at the left margin in the column with the first letter capitalized and the rest of the heading in lower case. Sub-sub-headings appear like sub-headings, except that they are in italics and not boldface. See examples in this file. No more than 3 levels of headings should be used.  Empty lines should be left above and below each section heading.

\subsection{Text font}

Times or Times New Roman font is used for the main text. Recommended font size is 11 points. Other font types may be used if needed for special purposes. If using any non-Unicode fonts, these must be embedded in the final PDF file.

The \LaTeX{} template can be used either with plain \LaTeX{} or \XeLaTeX.

\subsection{Figures}

All figures should be centered on the column (or page, if the figure
spans both columns). Figure captions should precede each figure and
have the format given in Fig.~\ref{fig:vowels}.

Figures should preferably be line drawings. If they contain grey
shades or colours, it should be checked that they print well on a
high-quality noncolour laser printer.

\begin{figure}[!ht]
\caption{The vowel chart used in the International Phonetic
Alphabet (IPA).}\label{fig:vowels}
\begin{center}
\includegraphics[width=6cm]{ipa.eps}
\end{center}
\end{figure}

\subsection{Tables}

An example of a table is shown as Table~\ref{tab:decibel}.  Somewhat
different styles are allowed according to the type and purpose of the
table. Colour should not be used, but grey shading is allowed. There
should be a margin of 6~points (pt) above and below the table.

The caption text may be above or below the table, but this should be
consistent throughout the submission. Left and right indentation of
the caption should be 0.5~cm.

\begin{table}[!ht]
\caption{This is an example of a table showing Decibel (dB)
ratios.}\label{tab:decibel}
\begin{center}
\begin{tabular}{|c|c|}
\hline
\rowcolor[gray]{.75}
ratio    & Decibels\\
\hline
$1/1$    & $0$\\
$2/1$    & $6$\\
$3.16$   & $10$\\
$1/10$   & $-20$\\
$10/1$   & $20$\\
$100/1$  & $40$\\
$1000/1$ & $60$\\
\hline
\end{tabular}
\end{center}
\end{table}


\subsection{Equations}

Equations should be placed on separate lines and numbered. An example
of an equation is given below:
\begin{equation}\label{eq:tzero}
t_0 = \frac{1}{f_0}
\end{equation}
Numbers of equations can be on the right or on the left margin of the
text column.

\subsection{Examples}

Examples from other languages can either be presented in the body text, or, if referred to elsewhere or particularly long and complex, can be put on a separate, numbered line, as should be done for equations.


\subsection{Phonetic fonts}

We recommend that you use Unicode IPA phonetic symbols. For information about how to access Unicode fonts, see \cite{IPA-SIL} or \cite{IPA-KEYBOARD}.  If you do not use Unicode symbols, the font you use must be embedded. Please remember to check this,  e.g. by inspecting the ``Document Properties --- Fonts'' in Acrobat Reader.


\subsection{Page numbering}

Page numbers will be added electronically to the document
later. \textit{Please do not add page numbers and please do not make
  any footers or headers!}

\subsection{References}

Please use just the reference number in square brackets. Formulations
with author names like ``\ldots\ as Ladefoged~\cite{Ladefoged:2003}
showed that \ldots'' are acceptable but not ``as shown in [Ladefoged,
7]'' or ``as shown in (Ladefoged~[7])''.

References are to be numbered in alphabetical order. Please
double-check the final version of your paper with regard to the
correct correspondence of references to their numbers.

\subsection{Hyperlinks}

Links to URLs or email addresses should be formatted as normal text,
\textit{not} as hyperlinks and not blue or underlined etc. Usually
hyperlinks to web pages are listed in the references section. If
required, line breaks can be placed within URLs after slashes or
dashes (cf.~\cite{IPA-SIL,IPA-KEYBOARD}), but doublecheck that no hyphens
are inserted.

\subsection{Footnotes and endnotes}

If footnotes cannot be avoided they should appear as
endnotes.\footnote{This footnote appears here as an endnote and could
be avoided in this case.}

\section{Multimedia files}

Multimedia data that are part of the paper are to be embedded in the submitted PDF; they cannot be submitted as supplementary data. Any images are to be included in the paper as Figures (see Section 2.3 above). It is the authors' responsibility to check image quality ahead of submission. Audio examples are to be embedded within the PDF. To do this, authors can generate the PDF, and then embed the audio files using Adobe Acrobat Professional. Alternatively, they may use other software that offers the same outcome, so that the audio is included in the PDF. The presence of audio data should be identified in the text.

We encourage authors to illustrate video data using still photographs from the video, and to include them as figures in the PDF.  We cannot accept embedded video files, but authors are welcome to refer readers to a URL on the internet where these can be viewed.


\section{PDF details}

PDF files submitted must comply with the following requirements:

\begin{enumerate}
\item all special fonts and symbols must be embedded in the PDF
  file so that correct rendering of the PDF does not depend on the
  fonts installed on the viewer's computer
\item there must be no
  password protection on the PDF file, i.e. PDF files must not be
  protected by PDF security in any way, i.e. content extraction,
  document assembly, high resolution printing etc. must not be
  forbidden
\item PDF files should not contain any colours, hyperlinks,
  multimedia or 3D content, and no JavaScript or forms
\item PDF files
  should be no larger than 5 Mb.
\end{enumerate}
\section{Anonymity}

In ICPhS 2015 submissions, an anonymous reviewing process will be used. This means that for the first submission the name(s) of the author(s) and
their affiliation(s) \emph{must not} be mentioned. In addition, please refrain from using acknowledgements.  Please also try to make your own previous research as anonymous as possible. As an example: do not write ``In our previous study [7] we could show\ldots'' but ``As shown in [7]\ldots''. Or refer to your own published or otherwise widely known work, and to that of the other authors, in the ``Julius Caesar style'', i.e. in the third person (for example: his work, her work, their work). Reference as ``anonymous'' only work that you or the other authors have submitted for publication, but which has not yet been published, e.g. [8].

Please make sure that no author details appear in the Document Properties of the PDF file. \emph{For the revised paper submission author details are of course needed}. Acknowledgements and references to one's own work are possible as usual.

\section{Format of references}

Monographs such as \cite{Fant:1960} consist of author(s) last name(s),
initial of the first name(s), year of publication, the title in
italics, location of the publication, publisher. Please use the
punctuation signs for structuring as presented in \cite{Fant:1960}.

The names of multiple authors are separated only by commas and they
are always listed in the sequence last name, comma, initial(s) of the
first name(s) (cf.\ the examples~\cite{Beattie/etal:1982} and
\cite{Peterson/Barney:1952}). Ampersands ``\&'' and ``and'' are not
needed.

Contributions to volumes, e.g.~\cite{Stevens:1999}, follow the
convention that the title of the volume is in italics, but not the
title of the contribution. The book editors appear before the book
title. The page numbers are at the end.

Journal articles should be handled in the same way as contributions to
volumes except that the title of the journal is in italics and that
the editors are not listed. Longer names of well-known journals can be
abbreviated, e.g.~\cite{Peterson/Barney:1952}. Articles in conference proceedings such as \cite{Ladefoged:2003} are
referenced in the same way as journal articles.  The word
\textit{proceedings} can be abbreviated and the location should be
mentioned after the name of the conference. Here, abbreviations of
well-known conferences are possible.

\bibliographystyle{icphs2015}
\bibliography{icphs2015}

\theendnotes

\end{document}
